\documentclass[oneside]{book}
\usepackage{amsthm}
\usepackage{a4wide}
\usepackage{booktabs}
\usepackage{microtype}
\usepackage{siunitx}
\usepackage{tikz}
\usepackage{pxfonts}
\usepackage{hyperref}
\usepackage{cleveref}
\usepackage{makeidx}
\usepackage{framed}
\usepackage{etoolbox}
\usepackage{comment}
\usepackage{ucllcode}

\includecomment{extra}
\specialcomment{example}{\vfill\begin{center}\begin{minipage}{.99\linewidth}\begin{framed}\begin{center}\begin{minipage}{.9\linewidth}\begin{center}\textsc{\bfseries Example}\end{center}}{\end{minipage}\end{center}\end{framed}\end{minipage}\end{center}\vfill}

\lstset{language=c++,basicstyle=\ttfamily,frame=lines}


\begin{document}

\begin{extra}
\tableofcontents
\end{extra}

\chapter{Primitive Types}
\begin{center}
  \begin{tabular}{ll}
    \textbf{Type} & \textbf{Description} \\
    \toprule
    \tt char & -128 to 127 \\
    \tt short & -32768 to 32767 \\
    \tt int & -2147483648 to 2147483647 \\
    \tt long & -2147483648 to 2147483647 \\
    \tt long long & -9223372036854775808 to 9223372036854775807 \\
    \midrule
    \tt unsigned char & 0 to 255 \\
    \tt unsigned short & 0 to 65535 \\
    \tt unsigned int & 0 to 4294967295 \\
    \tt unsigned long & 0 to 4294967295 \\
    \tt unsigned long long & 0 to 18446744073709551615 \\
    \midrule
    \tt float & $1.17549 \times 10^{-38}$ to $3.40282 \times 10^{38}$ \\
    \tt double & $2.22507 \times 10^{-308}$ to $1.79769 \times 10^{308}$ \\
    \midrule
    \tt bool & {\tt true}, {\tt false} \\
    \bottomrule
  \end{tabular}
\end{center}
Ranges were determined using MSVC++2013 targetting Win64.
\begin{extra}
  \code[font size=\small,width=.95\textwidth]{primtypes.cpp}
\end{extra}


\chapter{IO}

\begin{center}
  \begin{tabular}{ll}
    Standard input stream & \lstinline{std::cin} \\
    Standard output stream & \lstinline{std::cout} \\
    Reading from input stream & \lstinline{in >> var} \\
    Writing to output stream & \lstinline{out << var} \\
    Newline & \lstinline{std::endl} \\
  \end{tabular}
\end{center}

\begin{example}
  \code{hello-world.cpp}
\end{example}

\begin{example}
  \code{io-example.cpp}
\end{example}


\end{document}