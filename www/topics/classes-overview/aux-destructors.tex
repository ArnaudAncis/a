\section{Destructors}
\frame{\tableofcontents[currentsection]}

\begin{frame}
  \frametitle{Destructors}
  \begin{center}
    \begin{tabular}{lll}
      & \textbf{Constructor} & \textbf{Destructor} \\
      \toprule
      \textbf{Called} & at creation & on destruction \\
      \textbf{Name} & {\tt classname()} & {\tt \~{}classname()} \\
      \textbf{Parameters} & No restrictions & None \\
      \textbf{Overloads} & Possible & Impossible \\
    \end{tabular}
  \end{center}
  \vskip5mm
  \code[language=c++14,width=.6\linewidth]{destructor.cpp}
\end{frame}

\begin{frame}
  \frametitle{Reason for Destructors}
  \begin{itemize}
    \item Object requires resources (memory, files, \dots)
    \item When object dies, its resources need to be freed
          \begin{itemize}
            \item Heap allocated memory must be freed
            \item Files must be closed
            \item \dots
          \end{itemize}
    \item Constructor acquires resources
    \item Destructor frees those resources
  \end{itemize}
\end{frame}

\begin{frame}
  \frametitle{Destructors in Java}
  \begin{itemize}
    \item Java needs not free memory thanks to garbage collection
    \item Files/streams/\dots still need to be released
    \item Java went through several solutions
  \end{itemize}
\end{frame}

\begin{frame}
  \frametitle{Java Approach \#1: {\tt finalize}}
  \begin{itemize}
    \item Object has {\tt finalize} method ({\tt protected})
    \item Can be overriden in subclasses
    \item Called by garbage collector when object gets recycled
  \end{itemize}
  \code[language=java,font=\small]{finalize.java}
\end{frame}

\begin{frame}
  \frametitle{Problems with \texttt{finalize}}
  \begin{itemize}
    \item Nondeterministic: you never know when the GC will notice object has become unreachable
    \item {\tt finalize()} itself could make the object reachable again
          \begin{itemize}
            \item \link{https://en.wikipedia.org/wiki/Object_resurrection}{Zombie objects}
          \end{itemize}
    \item Rule of thumb: do not rely on {\tt finalize}
  \end{itemize}
  \code[language=java,font=\small]{zombie.java}
\end{frame}

\begin{frame}
  \frametitle{Java Approach \#2: {\tt try/finally}}
  \begin{itemize}
    \item Object has to be closed manually by calling {\tt close()}
    \item Should be used with {\tt try/finally}
  \end{itemize}
  \code[language=java,font=\small]{close.java}
\end{frame}

\begin{frame}
  \frametitle{Problems with {\tt try/finally}}
  \begin{itemize}
    \item Easy to forget
    \item Writing correct \texttt{try/finally} is actually complex
    \item Compiler does not remind you
    \item No consistent naming (e.g.\ {\tt close}, {\tt destroy}, {\tt dispose}, \dots)
  \end{itemize}
  \code[language=java,font=\tiny,width=.6\linewidth]{correct-close.java}
\end{frame}

\begin{frame}
  \frametitle{Java Approach \#3: {\tt Closeable}}
  \begin{itemize}
    \item Interface with {\tt close} method
    \item \link{https://docs.oracle.com/javase/tutorial/essential/exceptions/tryResourceClose.html}{\tt try-with-resources}
    \item Introduced in Java 7
  \end{itemize}
  \vskip5mm
  \code[language=java,font=\small,width=.9\linewidth]{try-with-resources.java}
\end{frame}

\begin{frame}
  \frametitle{\cpp\ Approach}
  \begin{itemize}
    \item Destructors are similar to {\tt finalize}
    \item Called when object gets destroyed
    \item Important difference: destructors are fully deterministic
  \end{itemize}
  \vskip5mm
  \begin{overprint}
    \onslide<handout:1|1>
    \code[language=c++14,width=.9\linewidth]{out-of-scope.cpp}

    \onslide<handout:2|2>
    \code[language=c++14,width=.9\linewidth]{delete.cpp}

    \onslide<handout:3|3>
    \code[language=c++14,width=.9\linewidth]{delete-array.cpp}

    \onslide<handout:4|4>
    \code[language=c++14,width=.9\linewidth]{pointer-out-of-scope.cpp}
  \end{overprint}
\end{frame}

\begin{frame}
  \frametitle{Usage: Example}
  \code[language=c++14]{destructor-usage.cpp}
\end{frame}

\begin{frame}
  \frametitle{Usage: Example}
  \code[language=c++14,width=.9\linewidth]{destructor-members.cpp}
  \begin{itemize}
    \item Destructor implicitly calls destructors of all fields
    \item A pointer's destructor does nothing!
  \end{itemize}
\end{frame}



\begin{frame}
  \frametitle{Destructors: Usage}
  \begin{itemize}
    \item Destructor needs to clean up things that do not clean up themselves (e.g.~pointers)
    \item Never call destructor yourself
  \end{itemize}
\end{frame}


%%% Local Variables:
%%% mode: latex
%%% TeX-master: "classes-overview"
%%% End:
