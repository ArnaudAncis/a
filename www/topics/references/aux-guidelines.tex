\section{Guidelines}

\frame{\tableofcontents[currentsection]}

\begin{frame}
  \frametitle{Limitations of References}
  \begin{itemize}
    \item References cannot be retargeted
          \begin{itemize}
            \item Once a reference refers to \texttt{x}, it does so forever
          \end{itemize}
    \item References cannot be null
  \end{itemize}
  \vskip4mm
  \code[language=c++14]{redirection.cpp}
\end{frame}

\begin{frame}
  \frametitle{When To Use References}
  \begin{itemize}
    \item Rule of Thumb: prefer references over pointers
          \begin{itemize}
            \item Unfortunately, pointers will often still be unavoidable
          \end{itemize}
    \item In practice: mostly used with functions
          \begin{itemize}
            \item \texttt{const T\&} parameters
            \item \texttt{T\&} return values
          \end{itemize}
  \end{itemize}
\end{frame}

\begin{frame}
  \frametitle{Do Not Use References for Out Parameters}
  \code[language=c++14]{out-params.cpp}
  \begin{itemize}
    \item It is important to know whether \texttt{x} could change
    \item If signature is \texttt{void foo(int)}, \texttt{x} cannot change
    \item If signature is \texttt{void foo(int\&)}, \texttt{x} could change
    \item Whether \texttt{x} can change depends on \texttt{foo}'s signature
    \item Signature not immediately visible above
    \item We would have to look it up
    \item Would seriously impede code reading
    \item Code above should visually express whether \texttt{x} can change
  \end{itemize}
\end{frame}

\begin{frame}
  \frametitle{Do Not Use References for Out Parameters}
  \begin{itemize}
    \item We introduce a convention
    \item Having to write \texttt{foo(x)} should mean \texttt{x} remain unmodified
          \begin{itemize}
            \item $\rightarrow$ \texttt{foo} receives \texttt{T} or \texttt{const T\&}
          \end{itemize}
    \item Having to write \texttt{foo(\&x)}, it signals \texttt{x} could change
          \begin{itemize}
            \item $\rightarrow$ \texttt{foo} receives \texttt{T*}
          \end{itemize}
    \item The \& makes it explicit that \texttt{x} could change
    \item No need to look up \texttt{foo}'s signature
    \item This way, we are able to understand code quicker
  \end{itemize}
\end{frame}

\begin{frame}
  \frametitle{How To Pass Parameters}
  \begin{center}
    \begin{tabular}{ccc}
      & \textbf{Small} & \textbf{Large} \\
      \toprule
      \textbf{Read} & \texttt{T} & \texttt{const T\&} \\
      \textbf{Read+Write} & \texttt{T*} & \texttt{T*} \\
    \end{tabular}
  \end{center}
  \vskip4mm
  \structure{Readonly Access}
  \begin{itemize}
    \item Small values: pass by value
    \item Large values: pass by const reference
  \end{itemize}
  \vskip4mm
  \structure{Read and Write Access}
  \begin{itemize}
    \item Always by pointer
  \end{itemize}
\end{frame}



%%% Local Variables:
%%% mode: latex
%%% TeX-master: "references"
%%% End:
