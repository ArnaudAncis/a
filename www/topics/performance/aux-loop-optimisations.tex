\section{Loop Optimisations}

\frame{\tableofcontents[currentsection]}

\begin{frame}
  \frametitle{Loop Optimisations}
  \begin{itemize}
    \item Loops can be rewritten in many ways
    \item Loops are perfect candidates for optimisation
          \begin{itemize}
            \item Most of execution time is spent in loops
          \end{itemize}
    \item Loops can be rewritten so as to
          \begin{itemize}
            \item Reduce loop overhead
            \item Improve locality
          \end{itemize}
  \end{itemize}
\end{frame}

\begin{frame}
  \frametitle{\link{https://en.wikipedia.org/wiki/Loop_unrolling}{Loop Unrolling}}
  \code[language=c++14,font=\small]{loop-unrolling.cpp}
  \vskip-5mm
  \begin{itemize}
    \item Reduces amount of loop condition checks
  \end{itemize}
\end{frame}

\begin{frame}
  \frametitle{\link{https://en.wikipedia.org/wiki/Loop_fission}{Loop Fission}}
  \code[language=c++14,font=\small]{loop-fission.cpp}
  \vskip-5mm
  \begin{itemize}
    \item Before: CPU hops back and forth in memory
    \item After: goes linearly over elements; better for cache
  \end{itemize}
\end{frame}

\begin{frame}
  \frametitle{\link{https://en.wikipedia.org/wiki/Loop_interchange}{Loop Interchange}}
  \code[language=c++14,font=\small]{loop-interchange.cpp}
  \vskip-5mm
  \begin{itemize}
    \item Before: CPU hops around a lot
    \item After: CPU processes each array sequentially
  \end{itemize}
\end{frame}

\begin{frame}
  \frametitle{\link{https://en.wikipedia.org/wiki/Strength_reduction}{Strength Reduction}}
  \code[language=c++14,font=\small]{strength-reduction.cpp}
  \vskip-5mm
  \begin{itemize}
    \item Rewriting so as to use less expensive operations
  \end{itemize}
\end{frame}



%%% Local Variables:
%%% mode: latex
%%% TeX-master: "performance"
%%% End:
