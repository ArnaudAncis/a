\usepackage{../pvm}
\usetikzlibrary{shadows,shapes.multipart}

\title{Casts}
\author{Fr\'ed\'eric Vogels}

\newcommand{\highlightbox}[2][]{
  \draw[opacity=.75,ultra thick,red,#1] ($ (#2.south west) + (-.1,-.1) $) rectangle ($ (#2.north east) + (.1,.1) $)
}


\begin{document}

\begin{frame}
  \titlepage
\end{frame}

\begin{frame}
  \frametitle{Casting}
  \begin{itemize}
    \item Casting: explicit type conversion
    \item Upcasting: casting to a supertype
          \begin{itemize}
            \item E.g.\ cast from {\tt Dog} to {\tt Animal}
          \end{itemize}
    \item Downcasting: casting to a subtype
          \begin{itemize}
            \item E.g.\ cast from {\tt Animal} to {\tt Dog}
          \end{itemize}
    \item Upcasts are always safe
    \item Downcasts can go wrong
  \end{itemize}
  \code{wrong-downcast.cpp}
\end{frame}

\begin{frame}
  \frametitle{Casts in Java}
  \begin{itemize}
    \item Only one way to cast: {\tt (T) x}
    \item Java objects keep track of their ``real type'' (dynamic type)
    \item Casts are checked at runtime
    \item Wrong casts: {\tt IllegalCastException}
  \end{itemize}
\end{frame}

\begin{frame}
  \frametitle{Casts in \cpp}
  \begin{itemize}
    \item \cpp~has a few more casts
          \begin{itemize}
            \item C-style cast: {\tt (T) x}
            \item {\tt static\_cast<T>(x)}
            \item {\tt dynamic\_cast<T>(x)}
            \item {\tt reinterpret\_cast<T>(x)}
            \item {\tt const\_cast<T>(x)}
          \end{itemize}
          \vskip5mm
    \item Standard Library adds a few more
          \begin{itemize}
            \item {\tt std::static\_pointer\_cast<T>(x)}
            \item {\tt std::dynamic\_pointer\_cast<T>(x)}
          \end{itemize}
          \vskip5mm
    \item Most important for us: {\tt dynamic\_pointer\_cast<T>(x)}
  \end{itemize}
\end{frame}

\section{C-style casts}

\frame{\tableofcontents[currentsection]}

\begin{frame}
  \frametitle{C-style Casts}
  \begin{itemize}
    \item Just don't use them
  \end{itemize}
\end{frame}

\section{Static and Dynamic Casts}

\frame{\tableofcontents[currentsection]}


\begin{frame}
  \frametitle{{\tt static\_cast} and {\tt dynamic\_cast}}
  \structure{{\tt static\_cast<T>(x)}}
  \begin{itemize}
    \item Only checked at compile time
    \item Wrong casts lead to undefined results at runtime
  \end{itemize}
  \vskip5mm
  \structure{{\tt dynamic\_cast<T>(x)}}
  \begin{itemize}
    \item Checked at compile and runtime
    \item Wrong casts give {\tt nullptr}
    \item Requires RTII (overhead)
  \end{itemize}
\end{frame}

\begin{frame}
  \frametitle{Examples}
  \code[font size=\small]{static-vs-dynamic-casts1.cpp}
\end{frame}

\begin{frame}
  \frametitle{Examples}
  \code[font size=\small]{static-vs-dynamic-casts2.cpp}
\end{frame}

\begin{frame}
  \frametitle{Examples}
  \code[font size=\small]{static-vs-dynamic-casts3.cpp}
\end{frame}

\begin{frame}
  \frametitle{Examples}
  \code[font size=\small]{static-vs-dynamic-casts4.cpp}
\end{frame}

\begin{frame}
  \frametitle{Smart Pointers}
  \begin{itemize}
    \item We said not to use regular pointers {\tt T*}
    \item Use {\tt std::unique\_ptr} and {\tt std::shared\_ptr} instead
    \item Library offers specialised cast operators for {\tt std::shared\_ptr}
    \item \emph{No} specialised casts for {\tt std::unique\_ptr} (AFAICT)
  \end{itemize}
  \code[font size=\small,width=.9\linewidth]{shared-ptr-casts.cpp}
\end{frame}

\section{{\tt reinterpret\_cast}}

\frame{\tableofcontents[currentsection]}


\begin{frame}
  \frametitle{{\tt reinterpret\_cast}}
  \begin{itemize}
    \item Anything goes
    \item No compile time checks
    \item No runtime checks
    \item Bad casts lead to undefined behaviour
  \end{itemize}
  \code[font size=\small]{reinterpret_cast.cpp}
\end{frame}

\begin{frame}
  \frametitle{{\tt reinterpret\_cast}}
  \begin{itemize}
    \item Allows you to probe memory
    \item Breaks encapsulation
  \end{itemize}
  \begin{overprint}
    \onslide<handout:0|1>
    \code[font size=\small]{find-password1.cpp}
    \onslide<handout:0|2>
    \code[font size=\tiny,width=.9\linewidth]{find-password2.cpp}
    \onslide<handout:0|3>
    \code[font size=\tiny,width=.8\linewidth]{find-password3.cpp}
  \end{overprint}
\end{frame}

{
  \newcommand{\HEX}[1]{\texttt{0x#1}}
  \newcommand{\fatentry}[3]{\HEX{#1} & #2 & #3 \\}

  \begin{frame}
    \frametitle{{\tt reinterpret\_cast<T>(x): What's It Good For?}}
    \begin{center} \Large
      Remember this?
    \end{center}
    \begin{center}
      \small
      \begin{tabular}{ccl}
        \textbf{Offset} & \textbf{Length} & \textbf{Description} \\
        \toprule
        \fatentry{00}{3}{Instructions to jump to the bootstrap code}
        \fatentry{03}{8}{Name of the formatting OS}
        \fatentry{0B}{2}{Bytes per sector}
        \fatentry{0D}{1}{Sectors per cluster}
        \fatentry{0E}{2}{Reserved sectors from the start of the volume}
        \fatentry{10}{1}{Number of FAT copies}
        \fatentry{11}{2}{Number of possible root entries}
        \fatentry{13}{2}{Small number of sectors}
        % \fatentry{15}{1}{Media descriptor}
        % \fatentry{16}{2}{Sectors per FAT}
        % \fatentry{18}{2}{Sectors per track}
        % \fatentry{1A}{2}{Number of heads}
        % \fatentry{1C}{4}{Hidden sectors}
        % \fatentry{20}{4}{Large number of sectors}
        % \fatentry{24}{1}{Drive number}
        % \fatentry{25}{1}{Reserved}
        % \fatentry{26}{1}{Extended boot signature}
        % \fatentry{27}{4}{Volume serial number}
        % \fatentry{2B}{11}{Volume label}
        % \fatentry{36}{8}{File system type --- should be \texttt{FAT12}}
        % \fatentry{3E}{448}{Bootstrap code}
        % \fatentry{1FE}{2}{Boot sector signature --- should be \HEX{AA55}}
        $\vdots$ & $\vdots$ & $\;\vdots$ \\
        \bottomrule
      \end{tabular}
    \end{center}
  \end{frame}
}

\begin{frame}
  \frametitle{FAT12}
  \code{fat12.cpp}
\end{frame}

\begin{frame}
  \frametitle{FAT12}
  \code[font size=\small,width=\linewidth]{read-data.cpp}
\end{frame}

\begin{frame}
  \frametitle{{\tt reinterpret\_cast}}
   \begin{itemize}
    \item If you read raw bytes from
          \begin{itemize}
            \item Disk
            \item Network
            \item \dots
          \end{itemize}
    \item You can interpret raw data for what they represent
          \begin{itemize}
            \item FAT data tables
            \item IP address
            \item TCP/IP packets
            \item \dots
          \end{itemize}
  \end{itemize}
  \code[font size=\small]{reinterpret.cpp}
\end{frame}

\section{{\tt const\_cast}}

\frame{\tableofcontents[currentsection]}

\begin{frame}
  \frametitle{{\tt const\_cast}}
  \begin{itemize}
    \item Used to remove {\tt const} qualified
    \item Only used for old C libraries
    \item Just don't use it
  \end{itemize}
  \code[font size=\small,width=.95\linewidth]{const-cast.cpp}
\end{frame}

\end{document}


%%% Local Variables:
%%% mode: latex
%%% TeX-master: "casts"
%%% End:
