\section{Examples}
\subsection{Indexing}
\frame{\tableofcontents[currentsubsection]}

\begin{frame}
  \frametitle{\tt Indexing}
  \begin{itemize}
    \item Indexing \texttt{[]} only defined on pointers/arrays
    \item However, you can overload \texttt{[]} for your own containers
    \item For example, \texttt{std::vector} overloads \texttt{[]}
  \end{itemize}
  \code[font=\small]{vector.cpp}
\end{frame}

\subsection{\texttt{toString}}
\frame{\tableofcontents[currentsubsection]}

\begin{frame}
  \frametitle{\texttt{toString} in \cpp}
  \code{cout.cpp}
  \begin{itemize}
    \item How to make this work?
  \end{itemize}
\end{frame}

\begin{frame}
  \frametitle{\texttt{toString} in \cpp}
  \code[font=\small,width=.9\linewidth]{cout-sol.cpp}
  \begin{itemize}
    \item Solution: overload \texttt{<<} on \texttt{std::ostream}
  \end{itemize}
\end{frame}

\subsection{Assignment}
\frame{\tableofcontents[currentsubsection]}

\begin{frame}
  \frametitle{Assignment}
  \code{assignment.cpp}
  \begin{itemize}
    \item By default, assignment \emph{between objects} overwrites the fields of the left operand
          with the values of the fields of the right operand
    \item \texttt{x} and \texttt{y} are still distinct objects after assignment
  \end{itemize}
\end{frame}

\begin{frame}
  \frametitle{Assignment}
  \begin{itemize}
    \item Do not confuse with assignment in Java!
    \item Java assignment operates on pointers
  \end{itemize}
  \code[width=.9\linewidth]{assignment-comparison.cpp}
\end{frame}

\begin{frame}
  \frametitle{Assignment}
  \code{intlist-assignment.cpp}
  \begin{itemize}
    \item Default behaviour is not always what we want
    \item \texttt{xs} and \texttt{ys} would share same internal array
    \item We need \texttt{xs} and \texttt{ys} to have separate internal arrays
    \item We need to redefine \texttt{=} on \texttt{IntList}
  \end{itemize}
\end{frame}

\begin{frame}
  \frametitle{Assignment}
  \code[font=\small,width=.9\linewidth]{intlist-assignment-corrected.cpp}
\end{frame}

%%% Local Variables:
%%% mode: latex
%%% TeX-master: "operator-overloading"
%%% End:
