\begin{frame}
  \frametitle{Example}
  \begin{center}
    \begin{tikzpicture}[scale=.25,transform shape,remember picture]
      \draw (0,0) grid (40,1);
      \node[anchor=east,font=\Huge] at (0,0) {$\cdots$};
      \node[anchor=west,font=\Huge] at (40,0) {$\cdots$};

      \foreach \i in {0,...,39} {
        \node[anchor=west,rotate=90] at ($ (\i,1) + (.5,0) $) {
          \tikzmath{
            int \address;
            \address = hex(1000 + \i);
             print {\tt0x\address};
          }
        };
      }

      \only<2->{
        \draw[fill=red!50,opacity=.9] (7,0) rectangle ++(4,1);
      }
      \node (x) at (9,0.5) {
        \only<2-3>{5}\only<4->{6}
      };

      \only<3->{
        \draw[fill=red!50,opacity=.9] (20,0) rectangle ++(4,1);
      }
      \node (p) at (22,0.5) {\only<3->{\texttt{0x3ef}}};
    \end{tikzpicture}
  \end{center}
  \vskip5mm
  \code[frame=lines,language=c++14]{example1.cpp}
  \begin{overprint}
    \onslide<2>
    \begin{center}
      \texttt{x} is allocated somewhere in memory \\
      (we ignore the static/stack/heap stuff)
    \end{center}
    \codeunderlinex{example 1 line 1}

    \onslide<3>
    \begin{center}
      \texttt{p} is allocated somewhere in memory \\
      It is initialised with \texttt{x}'s address \texttt{0x3ef}
    \end{center}
    \codeunderlinex{example 1 line 2}

    \onslide<4>
    \begin{center}
      We increment \texttt{x}
    \end{center}
    \codeunderlinex{example 1 line 3}

    \onslide<5>
    \begin{center}
      We print the value that \texttt{p} points at, i.e.~we
      read \texttt{p}, get an address, and then read what lies at this address
    \end{center}

    \codeoverlinex{example 1 dereference}
    \begin{tikzpicture}[overlay,remember picture]
      \draw[-latex,thick] (example 1 dereference) -- (p.south);
      \draw[-latex,thick] (p.west) -- (x.east);
    \end{tikzpicture}

    \onslide<6>
    \begin{center}
      Output: 6
    \end{center}
  \end{overprint}
\end{frame}

\begin{frame}
  \frametitle{Example}
  \begin{center}
    \begin{tikzpicture}[scale=.25,transform shape,remember picture]
      \draw (0,0) grid (40,1);
      \node[anchor=east,font=\Huge] at (0,0) {$\cdots$};
      \node[anchor=west,font=\Huge] at (40,0) {$\cdots$};

      \foreach \i in {0,...,39} {
        \node[anchor=west,rotate=90] at ($ (\i,1) + (.5,0) $) {
          \tikzmath{
            int \address;
            \address = hex(2100 + \i);
             print {\tt0x\address};
          }
        };
      }

      \only<2->{
        \draw[fill=red!50,opacity=.9] (19,0) rectangle ++(8,1);
      }
      \node (x) at (23,0.5) {
        \only<2-3>{18.0}\only<4->{36.0}
      };

      \only<3->{
        \draw[fill=red!50,opacity=.9] (2,0) rectangle ++(4,1);
      }
      \node (p) at (4,0.5) {\only<3->{\texttt{0x847}}};
    \end{tikzpicture}
  \end{center}
  \vskip5mm
  \code[frame=lines,language=c++14]{example2.cpp}
  \begin{overprint}
    \onslide<2>
    \begin{center}
      \texttt{x} is allocated somewhere in memory \\
    \end{center}

    \onslide<3>
    \begin{center}
      \texttt{p} is allocated somewhere in memory \\
      It is initialised with \texttt{x}'s address \texttt{0x847}
    \end{center}

    \onslide<4>
    \begin{center}
      We double the value pointed to by \texttt{p}, i.e.~we read \texttt{p},
      find out it points to \texttt{x}, and double \texttt{x}
    \end{center}

    \onslide<5>
    \begin{center}
      Output: 36
    \end{center}
  \end{overprint}
\end{frame}

\begin{frame}
  \frametitle{Pointers}
  \code[language=c++14,width=.4\linewidth]{equiv.cpp}
  \begin{itemize}
    \item \texttt{*p} signifies ``value pointed to by \texttt{p}''
    \item \texttt{*p} is just another way to mean \texttt{x}
    \item E.g.~\texttt{x++} is equivalent to \texttt{(*p)++}
  \end{itemize}
\end{frame}

\begin{frame}
  \frametitle{Example}
  \code[frame=lines,language=c++14]{example3.cpp}
  \visible<2>{
    \begin{center}
      Compiler error: cannot convert {\tt bool*} to {\tt int*}
    \end{center}
    \codeunderlinex{example 3 line 2}
  }
\end{frame}

\begin{frame}
  \frametitle{Example}
  \code[frame=lines,language=c++14]{example4.cpp}
  \begin{itemize}
    \item Initializer pointer with 5
    \item \texttt{*p = 1} writes one to memory location with address 5
    \item Allows you to write to random parts of memory
    \item Can overwrite other data structures
    \item Just don't do it
  \end{itemize}
\end{frame}

\begin{frame}
  \frametitle{Example}
  \begin{center}
    \begin{tikzpicture}[scale=.25,transform shape,remember picture]
      \draw (0,0) grid (40,1);
      \node[anchor=east,font=\Huge] at (0,0) {$\cdots$};
      \node[anchor=west,font=\Huge] at (40,0) {$\cdots$};

      \foreach \i in {0,...,39} {
        \node[anchor=west,rotate=90] at ($ (\i,1) + (.5,0) $) {
          \tikzmath{
            int \address;
            \address = hex(4500 + \i);
             print {\tt0x\address};
          }
        };
      }

      \draw[fill=red!50,opacity=.9] (1,0) rectangle ++(4,1);
      \node (x) at (3,0.5) {5};
      \draw[|-|] (1,-0.5) -- ++(4,0) node[midway,below] {x};
      
      \draw[fill=red!50,opacity=.9] (7,0) rectangle ++(4,1);
      \node (p) at (9,0.5) {\texttt{0x1195}};
      \draw[|-|] (7,-0.5) -- ++(4,0) node[midway,below] {p};

      \draw[fill=red!50,opacity=.9] (14,0) rectangle ++(4,1);
      \node (p) at (16,0.5) {\texttt{0x119b}};
      \draw[|-|] (14,-0.5) -- ++(4,0) node[midway,below] {q};

      \draw[fill=red!50,opacity=.9] (27,0) rectangle ++(4,1);
      \node (p) at (29,0.5) {\texttt{0x11a2}};
      \draw[|-|] (27,-0.5) -- ++(4,0) node[midway,below] {r};
    \end{tikzpicture}
  \end{center}
  \vskip5mm
  \code[frame=lines,language=c++14]{example5.cpp}
  \begin{center}
    \texttt{++***r}
  \end{center}
\end{frame}



%%% Local Variables:
%%% mode: latex
%%% TeX-master: "pointers"
%%% End:
