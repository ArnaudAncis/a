\usepackage{ucll-code}

\usetikzlibrary{shadows,shapes.multipart}

\title{Pointers}
\author{Fr\'ed\'eric Vogels}


\begin{document}

\begin{frame}
  \titlepage
\end{frame}


\input{aux-refs.tex}
\input{aux-intro.tex}
\section{Examples}
\frame{\tableofcontents[currentsection]}

\begin{frame}
  \frametitle{\tt Indexing}
  \begin{itemize}
    \item Indexing \texttt{[]} only defined on pointers/arrays
    \item However, you can overload \texttt{[]} for your own containers
    \item For example, \texttt{std::vector} overloads \texttt{[]}
  \end{itemize}
  \code[font=\small]{vector.cpp}
\end{frame}

\begin{frame}
  \frametitle{\texttt{toString} in \cpp}
  \code{cout.cpp}
  \begin{itemize}
    \item How to make this work?
  \end{itemize}
\end{frame}

\begin{frame}
  \frametitle{\texttt{toString} in \cpp}
  \code[font=\small,width=.9\linewidth]{cout-sol.cpp}
  \begin{itemize}
    \item Solution: overload \texttt{<<} on \texttt{std::ostream}
  \end{itemize}
\end{frame}


%%% Local Variables:
%%% mode: latex
%%% TeX-master: "operator-overloading"
%%% End:


\begin{frame}
  \frametitle{Summary}
  \begin{center}
    \begin{tabular}{ll}
      \textbf{Syntax}  & \textbf{Description} \\
      \toprule
      {\it type}*      & Pointer to a {\it type} \\
      \&{\it variable} & Address of {\it variable} \\
      *{\it pointer}   & Dereference pointer \\
    \end{tabular}
  \end{center}
  \structure{Typical Usage}
  \begin{enumerate}
    \item Introduce some variable \texttt{x} of type \texttt{T}
    \item Take the address of \texttt{x} using \texttt{\&x}
    \item Save the address of \texttt{x} in a pointer variable \texttt{p} of type \texttt{T*}
    \item You can access \texttt{x} using \texttt{*p}
  \end{enumerate}
\end{frame}

\begin{frame}
  \frametitle{Summary: Use Cases}
  \structure{Use Pointers When\dots}
  \begin{itemize}
    \item You need to pass large objects
    \item When you want to give access to your objects
  \end{itemize}
\end{frame}

\end{document}


%%% Local Variables:
%%% mode: latex
%%% TeX-master: "pointers"
%%% End:
