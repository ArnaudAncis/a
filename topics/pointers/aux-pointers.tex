\usepackage{../pvm}

\usetikzlibrary{shadows,shapes.multipart}

\title{Pointers}
\author{Fr\'ed\'eric Vogels}


\newcommand{\drawboxaround}[1]{
  \draw[box] ($ (#1.south west) + (-.15,-.15) $) rectangle ($ (#1.north east) + (.15,.15) $);
}


\begin{document}

\begin{frame}
  \titlepage
\end{frame}

\begin{frame}
  \frametitle{Pointers}
  \begin{itemize}
    \item Pointers have a reputation
    \item Pointers are \emph{not} hard
    \item If you can work with arrays in Java, you can work with pointers in \cpp
    \item Pointers are \emph{fragile}
    \item Easy to make mistakes
    \item But that's ok, because they're really simple
  \end{itemize}
\end{frame}

\begin{frame}
  \frametitle{Memory: A Simple Model}
  \begin{itemize}
    \item Memory can be seen as one gigantic {\tt byte[]} array
    \item Every time you need memory (e.g.\ a stack-variable, a heap-object, \dots)
          a little part of this array is given to you
    \item You could assign ranges to the different allocation methods, e.g.
          \begin{itemize}
            \item Static allocation: first 1000000 bytes
            \item Stack: following 1000000 bytes
            \item Heap: all the rest
          \end{itemize}
          but making this distinction not important right now
  \end{itemize}
\end{frame}

\begin{frame}
  \frametitle{Memory as a {\tt byte[]}}
  \begin{center}
    \begin{tikzpicture}[scale=.25,transform shape]
      \only<2->{
        \draw[fill=red!50] (7,0) rectangle ++(4,1);
      }

      \visible<3>{
        \foreach \i in {0,...,39} {
          \node[anchor=west,rotate=90] at ($ (\i,1) + (.5,0) $) {
            \tikzmath{
              int \address;
              \address = 4684 + \i;
               print {\tt\address};
            }
          };
        }
      }

      \visible<4->{
        \foreach \i in {0,...,39} {
          \node[anchor=west,rotate=90] at ($ (\i,1) + (.5,0) $) {
            \tikzmath{
              int \address;
              \address = hex(4684 + \i);
               print {\tt0x\address};
            }
          };
        }
      }

      \draw (0,0) grid (40,1);
      \node[anchor=east,font=\Huge] at (0,0) {$\cdots$};
      \node[anchor=west,font=\Huge] at (40,0) {$\cdots$};
    \end{tikzpicture}
  \end{center}
  \begin{itemize}
    \item If you need an {\tt int x} variable\dots
    \item \dots somewhere four consecutive bytes will be chosen,
          and will be associated with {\tt x}
    \item<3-> Since memory is a {\tt byte[]}, every byte has its unique index
    \item<4-> We can write this index in hexadecimal
    \item<5-> In the above figure: {\tt x} occupies bytes with indices {\tt 0x1256}, {\tt 0x1257}, {\tt 0x1258}, {\tt 0x1259}
    \item<5-> The index of the first byte of {\tt x} is the \emph{address} of {\tt x}
    \item<5-> Address of {\tt x} is {\tt 0x1256}
  \end{itemize}
\end{frame}

\begin{frame}
  \frametitle{A New Type}
  \begin{itemize}
    \item Given a variable {\tt int x}
    \item {\tt x} must have an address
    \item You can ask for this address
    \item New type: ``address of {\tt int}''
  \end{itemize}
  \vskip1cm
  \code[frame=lines,font size=\small]{address-of.cpp}
\end{frame}

\begin{frame}
  \frametitle{What Does One Do With An Address?}
  \begin{itemize}
  \item If you know where a variable resides in memory
        (i.e.\ if you have its address), then you can read and write to it
  \end{itemize}
  \vskip1cm
  \code[frame=lines,font size=\small]{use-address.cpp}
\end{frame}

\begin{frame}
  \frametitle{Quiz}
  \begin{itemize}
    \item Integer division {\tt x / y} has two results:
          \begin{itemize}
            \item Quotient {\tt q}
            \item Rest {\tt r}
          \end{itemize}
    \item {\tt q * y + r = x}
    \item How to write a function {\tt div} that returns quotient and rest?
  \end{itemize}
\end{frame}

\begin{frame}
  \frametitle{Div in Java: Attempt \#1}
  \code[language=java,font size=\small,frame=lines]{div1.java}
  \vskip5mm
  \structure{Evaluation}
  \begin{procontralist}
    \con<2-> Need for array creation on heap (slow)
    \con<2-> Array must contain 2 items, not compiler-enforced
    \con<2-> Works only if function returns values of same type
    \con<2-> No naming (is {\tt q} the first or second element?)
  \end{procontralist}
\end{frame}

\begin{frame}
  \frametitle{Div in Java: Attempt \#2}
  \code[language=java,font size=\small,frame=lines]{div2.java}
  \structure{Evaluation}
  \begin{procontralist}
    \pro<2-> Type safe
             \begin{itemize}
               \item \emph{Two} results enforced
               \item All type combinations possible
             \end{itemize}
    \pro<2-> Naming of results
    \con<2-> Need for object creation on heap (slow)
    \con<2-> Lots of ``stupid code'' necessary: we would need to define a new class for each function returning multiple values
  \end{procontralist}
\end{frame}

\begin{frame}
  \frametitle{Div in Java: Attempt \#3}
  \code[language=java,font size=\small,frame=lines]{div3.java}
  \structure{Evaluation}
  \begin{procontralist}
    \pro<2-> Type safe
    \pro<2-> Reusable {\tt Pair} class
    \con<2-> Need for three (!) object creations on heap
    \con<2-> No naming
    \con<2-> Boilerplate code alert: {\tt Triple}, {\tt Quadruple}, \dots
  \end{procontralist}
\end{frame}

\begin{frame}
  \frametitle{Div in Java: Attempt \#4}
  \code[language=java,font size=\small,frame=lines]{div4.java}
  \structure{Evaluation}
  \begin{procontralist}
    \pro<2-> Type safe
    \pro<2-> No allocations necessary
    \pro<2-> Clear naming
    \con<2-> It doesn't work \frownie
  \end{procontralist}
\end{frame}

\begin{frame}
  \frametitle{\cpp\ To The Rescue!}
  \code[width=\linewidth]{div.cpp}
\end{frame}

\begin{frame}
  \frametitle{Pass-by-value vs ``Pass-by-address''}
  \begin{itemize}
    \item Passing {\tt q} and {\tt r} as regular {\tt int}s does not work:
          copies are given to {\tt div} and no matter what {\tt div} does,
          it has no effect on {\tt func}'s {\tt q} and {\tt r}
    \item Passing the \emph{addresses} of {\tt q} and {\tt r}
          allows {\tt div} to access {\tt func}'s local variables:
          it knows where they live!
    \item This technique is called ``out parameters'': using parameters as output instead of input
  \end{itemize}
\end{frame}

\begin{frame}
  \frametitle{Uses}
  \begin{itemize}
    \item Using addresses has many other uses besides out parameters
    \item Java uses addresses all the time when dealing with objects
    \item More flexible (see later)
    \item More efficient (see later)
  \end{itemize}
\end{frame}

\begin{frame}
  \frametitle{Syntax}
  \structure{Pseudosyntax}
  \code[frame=lines]{pointer-syntax.cpp}
  \vskip5mm
  \structure{\cpp\ syntax}
  \code[frame=lines,width=.4\linewidth]{pointer-syntax2.cpp}

  \begin{tikzpicture}[overlay,remember picture,box/.style={red,thick}]
    \only<2>{
      \drawboxaround{address of int}
      \drawboxaround{int pointer}
    }
    \only<3>{
      \drawboxaround{address of}
      \drawboxaround{address of operator}
    }
    \only<4>{
      \drawboxaround{write to}
      \drawboxaround{write to syntax}
    }
  \end{tikzpicture}
\end{frame}

\begin{frame}
  \frametitle{Syntax}
  \begin{center}
    \begin{tabular}{ll}
      \textbf{Syntax}  & \textbf{Description} \\
      \toprule
      {\it type}*      & Pointer to a {\it type} \\
      \&{\it variable} & Address of {\it variable} \\
      *{\it pointer}   & Dereference pointer \\
    \end{tabular}
  \end{center}
  \begin{itemize}
    \item A \emph{pointer} is a variable that contains an address
    \item To access whatever is located at the address, you \emph{dereference} the pointer
    \item The {\tt \&} and {\tt *} need to be balanced
    \item Similar to arrays: given a {\tt T[] xs} you need to index {\tt xs} to reach a {\tt T}
  \end{itemize}
\end{frame}

\begin{frame}
  \frametitle{Quiz}
  \code[frame=lines]{quiz1.cpp}
  \begin{center}
    \visible<2>{6}
  \end{center}
\end{frame}

\begin{frame}
  \frametitle{Quiz}
  \code[frame=lines]{quiz2.cpp}
  \begin{center}
    \visible<2>{10}
  \end{center}
\end{frame}

\begin{frame}
  \frametitle{Quiz}
  \code[frame=lines]{quiz3.cpp}
  \begin{center}
    \visible<2>{Compiler error: cannot convert {\tt bool*} to {\tt int*}}
  \end{center}
\end{frame}

\begin{frame}
  \frametitle{Quiz}
  \code[frame=lines]{quiz4.cpp}
  \begin{center}
    \visible<2>{Undefined behaviour at runtime}
  \end{center}
\end{frame}

\begin{frame}
  \frametitle{Quiz}
  \code[frame=lines]{quiz5.cpp}
  \begin{center}
    \visible<2>{\tt ++***r;}
  \end{center}
\end{frame}


% Pass by value/pass by reference

\end{document}