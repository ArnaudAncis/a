\documentclass{../ucll-slides}
\usepackage{../pvm}
\usepackage{siunitx}
\usepackage{ifthen}

\usetikzlibrary{positioning}

\makeatletter
\def\light@caption@on{on}
\def\light@caption@off{off}
\def\light@caption@neutral{}
\makeatother

\tikzset{
  light/.pic={\draw node[circle,#1,minimum size=7mm,font=\tiny] {\csname light@caption@#1\endcsname};},
  light/.style={draw},
  on/.style={light,fill=yellow},
  off/.style={light,fill=gray!50},
  neutral/.style={light}
}

\title{Data Compression: Theory}
\author{Fr\'ed\'eric Vogels}

\begin{document}

\begin{frame}
  \titlepage
\end{frame}

\begin{frame}
  \frametitle{Working Example Contet}
  \begin{itemize}
    \item Say Alice and Bob live on separate islands
    \item Alice has a lighthouse on her island
    \item Alice can turn the light on or off, but only once
    \item Bob can see the lighthouse from his island
  \end{itemize}
\end{frame}

\begin{frame}
  \frametitle{Alice's Exam}
  \begin{itemize}
    \item Alice has had her exam PVM
    \item She wishes to inform Bob of the results
    \item She only wishes to tell him whether she passed or failed (i.e.\ not the exact grade)
    \item Can she use the lighthouse to inform Bob?
    \item (You can assume Alice and Bob were able to somehow agree beforehand on what an off/on light means)
  \end{itemize}
\end{frame}

\begin{frame}
  \frametitle{Alice's Exam}
  \structure{Possible Convention \#1}
  \begin{center}
    \begin{tabular}{m{1cm}m{2cm}}
      \tikz \pic {light=on}; & Passed \\
      \tikz \pic {light=off}; & Failed \\
    \end{tabular}
  \end{center}
  \vskip5mm
  \structure{Possible Convention \#2}
  \begin{center}
    \begin{tabular}{m{1cm}m{2cm}}
      \tikz \pic {light=off}; & Passed \\
      \tikz \pic {light=on}; & Failed \\
    \end{tabular}
  \end{center}
  \vskip5mm
  \begin{center}
    Which encoding is used is not important,
    what matters is that it's possible
  \end{center}
\end{frame}

\begin{frame}
  \frametitle{Information vs Encoding}
  \begin{center}
    \begin{tabular}{c@{\hspace{2cm}}c}
      \textbf{Information} & \textbf{Encoding} \\
      \toprule
      Passed & \alt<2>{1}{Light on} \\
      Failed & \alt<2>{0}{Light off} \\
    \end{tabular}
  \end{center}
  \vskip5mm
  \begin{itemize}
    \item Two possible ``informations''
    \item Two possible codes
    \item We can map each information to a code
    \item No ambiguities
    \item Injection
  \end{itemize}
\end{frame}

\begin{frame}
  \frametitle{Alice's Exam}
  \begin{itemize}
    \item Say Alice wants to be able to tell Bob whether she
      \begin{itemize}
        \item Passed
        \item Failed, but tolerable
        \item Failed, intolerable
      \end{itemize}
    \item Can she use the lighthouse to inform Bob?
  \end{itemize}
\end{frame}

\begin{frame}
  \frametitle{Information vs Encoding}
  \begin{center}
    \begin{tabular}{c@{\hspace{2cm}}c}
      \textbf{Information} & \textbf{Encoding} \\
      \toprule
      Passed & Light on (1) \\
      Tolerable & Light off (0) \\
      Intolerable
    \end{tabular}
  \end{center}
  \vskip5mm
  \begin{itemize}
    \item Three possible ``informations''
    \item Two possible codes
    \item We cannot map each information to a code
  \end{itemize}
  \vskip5mm
  \begin{center}
    \Large
    How to solve this problem? \\
    \visible<2>{Build a second lighttower!}
  \end{center}
\end{frame}

\begin{frame}
  \frametitle{Using Two Lighttowers}
  \structure{Possible Encoding}
  \begin{center}
    \begin{tabular}{m{2cm}m{2cm}}
      \tikz \pic {light=on}; \space \tikz \pic {light=on}; & Passed \\
      \tikz \pic {light=on}; \space \tikz \pic {light=off}; & Tolerable \\
      \tikz \pic {light=off}; \space \tikz \pic {light=off}; & Intolerable \\
    \end{tabular}
  \end{center}
  \begin{itemize}
    \item 24 possible encodings
    \item Again, not important which encoding
  \end{itemize}
\end{frame}

\begin{frame}
  \frametitle{Information vs Encoding}
  \begin{center}
    \begin{tabular}{c@{\hspace{2cm}}c}
      \textbf{Information} & \textbf{Encoding} \\
      \toprule
      Passed & 11 \\
      Tolerable & 10 \\
      Intolerable & 00
    \end{tabular}
  \end{center}
  \vskip5mm
  \begin{itemize}
    \item Three possible ``informations''
    \item Four possible codes
    \item Mapping possible
    \item Unused code 01
  \end{itemize}
\end{frame}

\begin{frame}
  \frametitle{Redundancy}
  \begin{itemize}
    \item Two lighttowers allow 4 codes
    \item Yet we only need 3
    \item Seems kind of wasteful
    \item We clearly need more than one lighthouse
    \item We clearly don't need two
    \item Build 1.5 lighthouses?
    \item Only solution: accept redundancy, build 2 lighthouses
  \end{itemize}
\end{frame}

\begin{frame}
  \frametitle{Multiple Exams}
  \begin{itemize}
    \item What if Alice has multiple exams, say 5
    \item She wishes to inform Bob about each exam whether she passed, failed tolerable or failed intolerably
    \item She needed 2 lighthouses per exam
    \item So she builds 10 lighthouses for 5 exams
  \end{itemize}
  \vskip5mm
  \[
    \foreach \x in {1,2,...,5} {
      \underbrace{
        \tikz\pic{light=neutral};
        \;
        \tikz\pic{light=neutral};
      }_{\textrm{exam \x}} \quad
    }
  \]
\end{frame}

\begin{frame}
  \frametitle{Information vs Encoding}
  \begin{center}
    \begin{tabular}{c@{\hspace{1cm}}c}
      \textbf{Information} & \textbf{Encoding} \\
      \toprule
      Exam 1: P/T/F & Lighthouse 1: on/off \\
      Exam 2: P/T/F & Lighthouse 2: on/off \\
      Exam 3: P/T/F & Lighthouse 3: on/off \\
      Exam 4: P/T/F & Lighthouse 4: on/off \\
      Exam 5: P/T/F & Lighthouse 5: on/off \\
                    & Lighthouse 6: on/off \\
                    & Lighthouse 7: on/off \\
                    & Lighthouse 8: on/off \\
                    & Lighthouse 9: on/off \\
                    & Lighthouse 10: on/off \\
      \midrule
      $3^5 = 243$ outcomes & $2^{10} = 1024$ codes
    \end{tabular}
  \end{center}
\end{frame}

\begin{frame}
  \frametitle{Redundancy}
  \begin{center}
    243 outcomes vs 1024 codes
  \end{center}
  \begin{itemize}
    \item Need a unique code for each of the 243 outcomes
    \item Is certainly possible: there are 1024 codes available
    \item But we're wasting 1024-243=781 codes
    \item Can we get away with less lighthouses?
  \end{itemize}
\end{frame}

\begin{frame}
  \frametitle{Number of Codes}
  \begin{center}
    \begin{tabular}{cc}
      \textbf{\#lighthouses} & \textbf{\#codes} \\
      \toprule
      1 & 2 \\
      2 & 4 \\
      3 & 8 \\
      4 & 16 \\
      5 & 32 \\
      6 & 64 \\
      7 & 128 \\
      \bf 8 & \bf 256 \\
      9 & 512 \\
      10 & 1024 \\
      \bottomrule
    \end{tabular}
  \end{center}
  \begin{itemize}
    \item We need 243 codes
    \item We can get away with 8 lighthouses
    \item How? \cake
  \end{itemize}
\end{frame}

\begin{frame}
  \frametitle{Compression}
  \begin{itemize}
    \item Original encoding used 10 lighthouses (bits)
    \item We really only need 8 lighthouses (bits)
    \item Finding the minimal number of bits needed = compression
  \end{itemize}
\end{frame}

\begin{frame}
  \frametitle{Bits vs Bits}
  \begin{itemize}
    \item You already know ``physical'' bits
          \begin{itemize}
            \item RAM
            \item Harddisk
            \item Network
            \item \dots
          \end{itemize}
    \item These are the bits used for the encoding
    \item If you have \SI{8}{GiB} storage, you have
          \[ 2^{2^{36}} \approx 10^{\SI{20000000000}{}} \]
          codes available
  \end{itemize}
\end{frame}

\begin{frame}
  \frametitle{Bits vs Bits}
  \begin{itemize}
    \item Distance can be measured in meters
    \item Time can be measured in seconds
    \item Weight can be measured in kilogram
    \item Information can also be quantified
    \item The unit for information is ``bit''
    \item To distinguish from the encoding bit, we'll use ``ibit''
    \item Formula to compute the number of ibits
          \[
            \log_2(\textrm{number of outcomes})
          \]
  \end{itemize}
\end{frame}

\begin{frame}
  \frametitle{Quantising Information: Example}
  \begin{itemize}
    \item One exam
    \item Two possible outcomes: Pass/Fail
  \end{itemize}
  \vskip5mm
  \[
    \log_2(2) = \SI{1}{ibits}
  \]
\end{frame}

\begin{frame}
  \frametitle{Quantising Information: Example}
  \begin{itemize}
    \item One exam
    \item Three possible outcomes: Pass/Tolerable/Intolerable
  \end{itemize}
  \vskip5mm
  \[
    \log_2(3) \approx \SI{1.58}{ibits}
  \]
  \vskip5mm
  \begin{itemize}
    \item Remember we needed more than one lighttower, but less than two
  \end{itemize}
\end{frame}

\begin{frame}
  \frametitle{Quantising Information: Example}
  \begin{itemize}
    \item Five exam
    \item Three possible outcomes: Pass/Tolerable/Intolerable
  \end{itemize}
  \vskip5mm
  \[
    \log_2(3^5) \approx \SI{7.92}{ibits}
  \]
  \vskip5mm
  \begin{itemize}
    \item Remember we needed 8 lighttowers
  \end{itemize}
\end{frame}

\begin{frame}
  \frametitle{Compression}
  \begin{itemize}
    \item If you have $N$ ibits of information\dots
    \item \dots you will minimally need $N$ bits of storage
    \item (Lossless) Compression = only using $N$ bits of storage
  \end{itemize}
\end{frame}

\begin{frame}
  \frametitle{Compression: Example}
  \begin{itemize}
    \item Five exams
    \item We want exact grade
    \item 0, 1, \dots, 20: 21 possible grades
    \item $\log_2(21^5) \approx \SI{21.96}{ibits}$
    \item You'd probably use an {\tt int} array: $5 \times 32 = \SI{160}{bits}$
    \item In reality, you only need \SI{22}{bits}
  \end{itemize}
\end{frame}

\begin{frame}
  \frametitle{Infinite Compression}
  \begin{itemize}
    \item Can you zip a zip?
    \item Can you continue compressing a file?
    \item Ibits form lower bound (if you want lossless)
    \item Lossy: no lower bound, but compressing every picture to a black image (only \SI{0}{bits} needed) is not very useful
  \end{itemize}
\end{frame}

\begin{frame}
  \frametitle{Infinite Compression}
  \begin{itemize}
    \item Say you invent a new compression algorithm ``zop''
    \item Say it compressed any $N$ bits file to a $N-1$ bits file.
    \item Any file can then be compressed to $1$ bit.
    \item How can you reconstruct a movie/music/data based on one single bit?
  \end{itemize}
  \begin{center}
    \begin{tikzpicture}
      \foreach \i in {1,...,8} {
        \tikzmath{
          real \size;
          \size = 0.85^\i;
          int \intensity;
          \intensity = int((1 - \size) * 100);
        }
        \node[circle,minimum size=\size cm,draw,inner sep=0mm,fill=red!\intensity] (d\i) at (\i,0) {};
      }

      \foreach[count=\j] \i in {2,...,8} {
        \draw[-latex] (d\j) -- (d\i) node[rotate=90,midway,font=\tiny,anchor=west] {zop!};
      }
    \end{tikzpicture}
  \end{center}
  \begin{itemize}
    \item Infinite compression is impossible.
  \end{itemize}
\end{frame}

\end{document}