\documentclass{../ucll-slides}
\usepackage{../pvm}
\usepackage{fourier}

\usetikzlibrary{shadows,shapes.multipart}

\title{Allocation Methods}
\author{Fr\'ed\'eric Vogels}



\begin{document}

\begin{frame}
  \titlepage
\end{frame}

\begin{frame}
  \frametitle{Overview}
  \begin{itemize}
    \item Different ways to deal with RAM
    \item Similar to data structures
    \item Trade flexibility for speed
    \item Normally abstracted away as much as possible
    \item In \cpp: crucial to know about the details
  \end{itemize}
\end{frame}

\section{Static Allocation}

\begin{frame}
  \tableofcontents[currentsection]
\end{frame}

\begin{frame}
  \frametitle{Static Allocation}
  \begin{itemize}
    \item Simplest allocation mechanism
    \item Fastest
    \item Most restrictive
    \item Used in oldest programming language \\ (Fortran, early 1950s)
  \end{itemize}
  \vskip1cm
  \begin{center}
    {\color{red}\Huge\danger}
    \parbox{8cm}{\centering 
      {\color{red}\Huge \textsc{warning}} \\
      Examples that follow are what-ifs \\
      They do not show how really \cpp\ works
    }
    {\color{red}\Huge\danger}
  \end{center}
\end{frame}

\begin{frame}
  \frametitle{How Does It Work?}
  \notrealcpp
  \begin{itemize}
    \item The compiler assigns a fixed address to each variable
  \end{itemize}
  \code{allocation.cpp}
  \begin{tikzpicture}[overlay,remember picture,allocation/.style={-latex,thick}]
    \coordinate (upperleft) at (6,4);

    \begin{scope}[scale=.5]
      \coordinate (double) at (upperleft);
      \coordinate (int) at ($ (upperleft) + (0,-1) $);
      \coordinate (char) at ($ (upperleft) + (4,-1) $);

      \visible<2->{
        \draw[fill=red!50] (double) rectangle ++(8,-1);
      }

      \visible<3->{
        \draw[fill=green!50] (int) rectangle ++(4,-1);
      }

      \visible<4->{
        \draw[fill=blue!50] (char) rectangle ++(1,-1);
      }

      \draw (upperleft) grid ++(8,-4);
    \end{scope}

    \visible<2>{
      \draw[allocation] (global) -- (double);
    }

    \visible<3>{
      \draw[allocation] (param) -- ($ (int) + (0,-0.25) $);
    }

    \visible<4>{
      \draw[allocation] (local) -- ($ (char) + (0,-0.25) $);
    }
  \end{tikzpicture}
\end{frame}

\begin{frame}
  \frametitle{Consequences of Static Allocation}
  \notrealcpp
  \begin{itemize}
    \item Compiler needs to have full knowledge at compile-time
  \end{itemize}
  \code{array-creation.cpp}
\end{frame}

\begin{frame}
  \frametitle{Consequences of Static Allocation}
  \notrealcpp
  \begin{itemize}
    \item Local variables ``remember'' their values
  \end{itemize}
  \code{static-locals.cpp}
\end{frame}

\begin{frame}
  \frametitle{Consequences of Static Allocation}
  \notrealcpp
  \begin{itemize}
    \item Recursion not possible
  \end{itemize}
  \code{static-recursion.cpp}
\end{frame}

\begin{frame}
  \frametitle{Consequences of Static Allocation}
  \begin{itemize}
    \item No out-of-memory errors possible
    \item Entire memory footprint known beforehand
    \item Full recompilation needed to allow larger data structures
    \item Cannot make use of extra RAM
      \begin{itemize}
        \item If it's been compiled to work with 1MB but you have 8GB, too bad
      \end{itemize}
  \end{itemize}
\end{frame}

\section{Stack Allocation}

\begin{frame}
  \tableofcontents[currentsection]
\end{frame}

\begin{frame}
  \frametitle{Stack Allocation}
  \notrealcpp
  \begin{columns}
    \column{5cm}
    \code[font size=\small]{stack-allocation.cpp}
    \column{4cm}
    \begin{center}
      \begin{tikzpicture}[allocation/.style={-latex,thick},remember picture,overlay,scale=.5]
        \coordinate (start) at (-4,4);

        \only<3->{
          \draw[fill=red!50] (start) rectangle ++(4,-1);
        }
        \only<3>{
          \draw[allocation] (var a) to [bend left=30] ($ (start) + (0,-0.5) $);
        }

        \only<4->{
          \draw[fill=green!50] ($ (start) + (4,0) $) rectangle ++(4,-1);
        }
        \only<4>{
          \draw[allocation] (var b) to [bend left=30] ($ (start) + (4,-0.5) $);
        }

        \only<5->{
          \draw[fill=blue!50] ($ (start) + (0,-1) $) rectangle ++(4,-1);
        }
        \only<5>{
          \draw[allocation] (var c) to [bend left=30] ($ (start) + (0,-1.5) $);
        }

        \only<6->{
          \draw[fill=yellow!50] ($ (start) + (4,-1) $) rectangle ++(4,-1);
        }
        \only<6>{
          \draw[allocation] (var d) to [bend left=30] ($ (start) + (4,-1.5) $);
        }

        \only<8->{
          \draw[fill=red!50] ($ (start) + (0,-2) $) rectangle ++(8,-2);
        }
        \only<8>{
          \draw[allocation] (mem a) to [bend right=30] ($ (start) + (0,-2.5) $);
        }

        \only<9->{
          \draw[fill=green!50] ($ (start) + (0,-4) $) rectangle ++(8,-3);
        }
        \only<9>{
          \draw[allocation] (mem b) to [bend right=30] ($ (start) + (0,-4.5) $);
        }

        \only<10->{
          \draw[fill=blue!50] ($ (start) + (0,-7) $) rectangle ++(4,-1);
        }
        \only<10>{
          \draw[allocation] (mem c) to [bend right=30] ($ (start) + (0,-7.5) $);
        }

        \only<11->{
          \draw[fill=yellow!50] ($ (start) + (4,-7) $) rectangle ++(4,-1);
          \draw[fill=yellow!50] ($ (start) + (0,-8) $) rectangle ++(8,-1);
        }
        \only<11>{
          \draw[allocation] (mem d) to [bend right=30] ($ (start) + (4,-7.5) $);
        }

        \draw (start) grid ++(8,-9);
      \end{tikzpicture}
    \end{center}
  \end{columns}
  \vskip2cm
  \begin{overprint}
    \onslide<2-6>
    \begin{center}
      \textbf{Compilation stage} \\ Compiler assigns memory locations to each variable \\ (this is still static allocation)
    \end{center}
    \onslide<7->
    \begin{center}
      \textbf{At runtime} \\ Extra memory is allocated on stack at runtime
    \end{center}
  \end{overprint}
\end{frame}


\end{document}