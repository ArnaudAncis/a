\documentclass{../ucll-slides}
\usepackage{../pvm}
\usetikzlibrary{shadows,shapes.multipart}

\title{Classes: Overview}
\author{Fr\'ed\'eric Vogels}

\newcommand{\highlightbox}[2][]{
  \draw[opacity=.75,ultra thick,red,#1] ($ (#2.south west) + (-.1,-.1) $) rectangle ($ (#2.north east) + (.1,.1) $)
}


\begin{document}

\begin{frame}
  \titlepage
\end{frame}

\begin{frame}
  \frametitle{Classes}
  \begin{itemize}
    \item Member variables (fields)
    \item Member functions (methods)
    \item Access modifiers ({\tt public}, {\tt protected}, {\tt private})
    \item Constructors
    \item \emph{Destructors}
  \end{itemize}
\end{frame}

\begin{frame}
  \frametitle{Classes: Basic Syntax}
  \code[frame=lines]{basic-syntax.cpp}
  \begin{tikzpicture}[overlay,remember picture]
    \only<2>{
      \highlightbox{private};
      \highlightbox{public};
    }
    \only<3>{
      \highlightbox{semicolon};
    }
  \end{tikzpicture}
  \begin{overprint}
    \onslide<2>
    \begin{itemize}
      \item Members with same access are grouped
      \item You can alternate as many times you you want
      \item Default: {\tt private}
    \end{itemize}
    \onslide<3>
    \begin{itemize}
      \item Class declaration ends with {\tt ;}
      \item Reason: you can put a list of identifiers there
    \end{itemize}
  \end{overprint}
\end{frame}

\begin{frame}
  \frametitle{Member Functions Inside Class Declaration}
  \code[frame=lines]{member-functions.cpp}
  \begin{itemize}
    \item Only for (very) short methods
  \end{itemize}
\end{frame}

\begin{frame}
  \frametitle{Member Functions Outside Class Declaration}
  \code[frame=lines]{member-functions2.cpp}
  \begin{tikzpicture}[overlay,remember picture]
    \only<2>{
      \highlightbox{class designator};
    }
  \end{tikzpicture}
  \begin{itemize}
    \item<2> Indicates the member function's class
  \end{itemize}
\end{frame}

\begin{frame}
  \frametitle{Constructors}
  \code[frame=lines,font size=\small]{constructors.cpp}
  \begin{tikzpicture}[overlay,remember picture,ref/.style={red,ultra thick,-latex,opacity=.75}]
    \only<2>{
      \draw[ref] (name field ref) -- (name field);
      \draw[ref] (name param ref) -- (name param);
    }
    \only<3>{
      \draw[ref] (age field ref) -- (age field);
      \draw[ref] (age param ref) -- (age param);
    }
  \end{tikzpicture}
  \begin{itemize}
    \item Make use of initializer list \cake
  \end{itemize}
\end{frame}

\begin{frame}
  \frametitle{Delegating To Other Constructor}
  \code[frame=lines]{construction-delegation.cpp}
\end{frame}

\begin{frame}
  \frametitle{Quick Intermezzo: Better Solution}
  \code[frame=lines]{default-arguments.cpp}
\end{frame}

\begin{frame}
  \frametitle{Instantiating Classes (Stack)}
  \code[frame=lines]{instantiatation-stack.cpp}
\end{frame}

\begin{frame}
  \frametitle{Instantiating Classes (Heap)}
  \begin{overprint}
    \onslide<1>
    \code[frame=lines]{instantiatation-heap.cpp}
    \onslide<2>
    \code[frame=lines]{instantiatation-heap2.cpp}
  \end{overprint}
\end{frame}

\begin{frame}
  \frametitle{Arrow Operator}
  \begin{itemize}
    \item Objects tend to be large (larger than register size)
    \item Generally passed around by pointer or by reference
    \item When by pointer: clumsy syntax ({\tt (*p).method()})
    \item Shorthand syntax: {\tt p->method()}
  \end{itemize}
\end{frame}

\begin{frame}
  \frametitle{Destructors}
  \begin{itemize}
    \item Constructor: called on creation
    \item Destructor: called on destruction
    \item Constructor name: same as class
    \item Destructor name: class name prefixed with \~{}
    \item Never accepts parameters
    \item Maximum one destructor allowed
  \end{itemize}
  \vskip5mm
  \code[frame=lines]{destructor.cpp}
\end{frame}

\begin{frame}
  \frametitle{Reason for Destructors}
  \begin{itemize}
    \item Object requires resources (memory, files, \dots)
    \item When object dies, resources need to be freed
    \item Constructor acquires resources
    \item Destructor frees resources
  \end{itemize}
\end{frame}

\begin{frame}
  \frametitle{Destructors in Java}
  \begin{itemize}
    \item Java needs not free memory thanks to garbage collection
    \item Files/streams/\dots needs to be released
    \item Java went through several solutions
  \end{itemize}
\end{frame}

\begin{frame}
  \frametitle{Java Approach \#1: {\tt finalize}}
  \begin{itemize}
    \item Object has {\tt finalize} method ({\tt protected})
    \item Can be overriden in subclasses
    \item Called when garbage collector sees object became unreachable
  \end{itemize}
  \vskip5mm
  \structure{Problems}
  \begin{itemize}
    \item Nondeterministic: you never know when the GC will notice object has become unreachable
    \item {\tt finalize()} itself could make the object reachable again
    \item Zombie objects
  \end{itemize}
\end{frame}

\begin{frame}
  \frametitle{Java Approach \#2: {\tt close}}
  \begin{itemize}
    \item Object has to be closed manually by calling {\tt close()}
    \item Should be used with {\tt try/finally}
  \end{itemize}
  \vskip5mm
  \structure{Problems}
  \begin{itemize}
    \item Easy to forget
    \item Compiler does not remind you
    \item Consistent naming not enforced (every class can use different name, e.g.\ {\tt close}, {\tt destroy}, {\tt dispose}, \dots)
  \end{itemize}
\end{frame}

\begin{frame}
  \frametitle{Java Approach \#3: {\tt Closeable}}
  \begin{itemize}
    \item Interface with {\tt close} method
    \item \link{https://docs.oracle.com/javase/tutorial/essential/exceptions/tryResourceClose.html}{\tt try-with-resources}
  \end{itemize}
  \vskip5mm
  \code[frame=lines,language=java,font size=\small]{try-with-resources.java}
\end{frame}

\begin{frame}
  \frametitle{\cpp\ Approach}
  \begin{itemize}
    \item Destructors are similar to {\tt finalize}
    \item Called when object gets destroyed
    \item Important difference: destructors are fully deterministic
  \end{itemize}
  \vskip5mm
  \begin{overprint}
    \onslide<1>
    \code[frame=lines,width=.9\linewidth]{out-of-scope.cpp}

    \onslide<2>
    \code[frame=lines,width=.9\linewidth]{delete.cpp}

    \onslide<3>
    \code[frame=lines,width=.9\linewidth]{delete-array.cpp}
  \end{overprint}
\end{frame}

% Const
% Static
% Creating objects
% Constructor
% Destructor

\end{document}