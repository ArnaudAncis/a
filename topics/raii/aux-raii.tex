\usepackage{../pvm}

\usetikzlibrary{shadows,shapes.multipart}

\title{RAII}
\author{Fr\'ed\'eric Vogels}


\begin{document}

\begin{frame}
  \titlepage
\end{frame}

\begin{frame}
  \frametitle{Resources}
  \structure{What are resources?}
  \begin{itemize}
    \item Can be acquired
    \item While acquired, can often not be shared
    \item Must be released so that others can use it
  \end{itemize}
  \vskip5mm
  \structure{Examples of resources}
  \begin{itemize}
    \item Memory
    \item Files
    \item Ports (networking)
    \item GPU (pre Windows Vista)
  \end{itemize}
\end{frame}

\begin{frame}
  \frametitle{Example: Heap Memory}
  \begin{itemize}
    \item Acquiring: {\tt T* p = new T}
    \item Releasing: {\tt delete p;}
    \item Sharing not allowed: two objects can never share memory
    \item Not releasing leads to memory leaks 
    \item Memory leaks lead to memory exhaustion
    \item Memory exhaustion leads to crash
    \item Crash leads to unhappy users
    \item You do not want unhappy users
  \end{itemize}
\end{frame}

\begin{frame}
  \frametitle{Example: Files}
  \begin{itemize}
    \item Sharing is possible
    \item Multiple readers
    \item Single writer
    \item Multiple writers: feasible but difficult
    \item Files must be ``closed''
    \item Consequences of not closing:
          \begin{itemize}
            \item Readers block writers
            \item Writers block readers and writers
          \end{itemize}
  \end{itemize}
\end{frame}

\begin{frame}
  \frametitle{Resources}
  \begin{itemize}
    \item Correct resource management is important
    \item Always free your resources
    \item Resource exhaustion is a Bad Thing$^{\textrm{TM}}$
    \item Keep exceptions in mind!
  \end{itemize}
  \vskip5mm
  \structure{Resource Management in an Ideal World}
  \begin{itemize}
    \item You want the resource to be freed \emph{automatically}
    \item You don't want to be able to forget to free resources
    \item You want to know when the resource gets freed
    \item You don't want to have take exceptions into account
  \end{itemize}
\end{frame}

\begin{frame}
  \frametitle{Java}
  \begin{itemize}
    \item {\tt finalize}
          \begin{procontralist}
            \pro Release cannot be forgotten
            \con Nondeterministic: you don't know when it will be called
            \con Trouble with zombies
            \con Discouraged
          \end{procontralist}
    \item {\tt close} in {\tt finally}
          \begin{procontralist}
            \pro Deterministic
            \con Difficult to get it right
            \con Lots of boilerplate
            \con You can still forget to release resources
          \end{procontralist}
    \item Since 1.7: \link{https://docs.oracle.com/javase/tutorial/essential/exceptions/tryResourceClose.html}{try with resources}
          \begin{procontralist}
            \pro Deterministic
            \pro No code necessary to release, hence cannot be forgotten
          \end{procontralist}
  \end{itemize}
\end{frame}

\begin{frame}
  \frametitle{\cpp}
  \begin{itemize}
    \item RAII: Resource Acquisition is Initialisation
    \item Resource is represented by object
    \item Object has destructor
    \item Destructor is responsible for releasing resource
  \end{itemize}
\end{frame}

\begin{frame}
  \frametitle{Explicit Release}
  \code[font size=\small]{explicit-delete.cpp}
  \begin{itemize}
    \item Horribly fragile
    \item What if {\tt ...} throws exception?
    \item What if multiple {\tt return}s? Multiple {\tt delete}s needed
  \end{itemize}
\end{frame}

\begin{frame}
  \code[font size=\small]{using-deleter.cpp}
\end{frame}

\begin{frame}
  \frametitle{\tt deleter}
  \begin{itemize}
    \item {\tt deleter} auxiliary objects
    \item Destructor will be called whenever it goes out of scope
          \begin{itemize}
            \item On function's end of body
            \item At every {\tt return}
            \item At every exception
          \end{itemize}
    \item Still need not to forget about {\tt deleter}
    \item Clumsy solution
  \end{itemize}
\end{frame}

\begin{frame}
  \code[font size=\small]{intarray.cpp}
\end{frame}

\begin{frame}
  \frametitle{Destructors}
  \begin{itemize}
    \item Every resource should be represented by an object
    \item The destructor should always release the resource
    \item No leaks possible
  \end{itemize}
\end{frame}

\begin{frame}
  \frametitle{Example: Memory Resource}
  \code[font size=\small]{vector.cpp}
  \begin{itemize}
    \item {\tt ns} allocates memory
    \item This memory is deallocated automatically by {\tt ns}'s destructor
    \item {\tt vector}s can be seen as ``automated resource management for heap allocated arrays''
  \end{itemize}
\end{frame}

\begin{frame}
  \frametitle{Example: File Resource}
  \code[font size=\small,width=.9\linewidth]{file.cpp}
\end{frame}

\end{document}


%%% Local Variables:
%%% mode: latex
%%% TeX-master: "raii"
%%% End:
