\begin{frame}
  \frametitle{Call By Reference}
  \code[frame=lines,width=.5\linewidth,font size=\small]{call-by-reference.cpp}
  \begin{itemize}
    \item {\tt foo} receives a \emph{reference} to a {\tt Bar}
    \item I.e.~{\tt foo} gets the address of a {\tt Bar} object
    \item {\tt foo} can read the object
          \begin{itemize}
            \item {\tt p} yields the {\tt Bar} object itself
            \item {\tt p.x} reads the {\tt x} member variable
          \end{itemize}
    \item {\tt foo} can also write to the object
          \begin{itemize}
            \item E.g.~{\tt p.x = 5}
          \end{itemize}
    \item {\tt foo} can do the same as with call-by-pointer
    \item Call-by-pointer and call-by-reference differ in syntax, not much more
  \end{itemize}
\end{frame}

\begin{frame}
  \frametitle{Call By Reference}
  \begin{center}
    \begin{columns}
      \column{6cm}
      \code[frame=lines,width=.95\linewidth,font size=\small]{call-by-reference.cpp}
      \column{6cm}
      \begin{tikzpicture}
        \memorylayout

        \only<2-6>{
          \stackframe[start=0,contents={bar.x = 1},id=bar]
        }
        \only<7->{
          \stackframe[start=0,contents={bar.x = 2},id=bar]
        }
        \only<3->{
          \stackframe[start=1,contents={bar.y = 2}]
        }
        \only<5-8>{
          \stackframe[start=2,contents={b},id=p]
          \draw[-latex] (p.east) to [bend right=30] (bar.east);
        }
      \end{tikzpicture}
    \end{columns}
  \end{center}
  \vskip2mm
  \begin{overprint}
    \onslide<1-3>
    \begin{center}
      Creation of bar on stack \\
      Default constructor does the job
    \end{center}

    \onslide<4-5>
    \begin{center}
      Calling {\tt foo}: {\tt bar} is passed by reference \\
      $\Rightarrow$ no copy is made \\
      $\Rightarrow$ no constructor gets called
    \end{center}

    \onslide<6-7>
    \begin{center}
      Incrementing {\tt b.x} operates on original
    \end{center}

    \onslide<8-9>
    \begin{center}
      Returning from {\tt foo}: all locals are cleaned up \\
      Only local variable is {\tt b}, a reference \\
      A reference's destructor does nothing
    \end{center}

    \onslide<10>
    \begin{center}
      Note how {\tt bar} has changed since before the call to {\tt foo}
    \end{center}
  \end{overprint}
\end{frame}



%%% Local Variables:
%%% mode: latex
%%% TeX-master: "technical-details"
%%% End:
