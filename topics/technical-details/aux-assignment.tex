\begin{frame}
  \frametitle{Assignment Operator}
  \code[width=.5\linewidth]{assignment.cpp}
  \begin{itemize}
    \item {\tt x = y} where {\tt x} and {\tt y} are objects (not pointers/references)
          copies the contents of {\tt y} to {\tt x}
    \item The {\tt =} operator will generally overwrite all of {\tt x}'s fields with {\tt y}'s field values.
    \item It is important to distinguish between using {\tt =} on objects and using {\tt =} on pointers
  \end{itemize}
\end{frame}

\begin{frame}
  \frametitle{Assignment Operator On Objects}
  \begin{center}
    \begin{columns}
      \column{5cm}
      \code[frame=lines,width=.95\linewidth,font=\small]{assignment.cpp}
      \column{4cm}
      \begin{tikzpicture}
        \memorylayout
        
        \only<2-7>{
          \stackframe[start=0,contents={bar.x = 1}]
        }
        \only<8->{
          \stackframe[start=0,contents={bar.x = 3}]
        }
        \only<3-8>{
          \stackframe[start=1,contents={bar.y = 2}]
        }
        \only<9->{
          \stackframe[start=1,contents={bar.y = 6}]
        }
        \only<5-11>{
          \stackframe[start=2,contents={temp.x = 1}]
        }
        \only<6-10>{
          \stackframe[start=3,contents={temp.y = 2}]
        }
      \end{tikzpicture}
    \end{columns}
  \end{center}
  \vskip2mm
  \begin{overprint}
    \onslide<1-3>
    \begin{center}
      {\tt bar} gets created on stack using the default constructor
    \end{center}

    \onslide<4-6>
    \begin{center}
      {\tt foo} gets called. It creates a new object (default constructor) which we'll call {\tt temp}.
    \end{center}

    \onslide<7-9>
    \begin{center}
      {\tt Bar::operator =} gets called on {\tt bar} with as argument a reference to {\tt temp}
    \end{center}

    \onslide<10-12>
    \begin{center}
      {\tt temp} gets destroyed (destructor gets called)
    \end{center}
  \end{overprint}
\end{frame}

%%% Local Variables:
%%% mode: latex
%%% TeX-master: "technical-details"
%%% End:
