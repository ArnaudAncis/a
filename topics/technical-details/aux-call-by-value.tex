\begin{frame}
  \frametitle{Call By Value}
  \begin{center}
    \begin{columns}
      \column{6cm}
      \code[frame=lines,width=.95\linewidth,font size=\small]{call-by-value.cpp}
      \column{6cm}
      \begin{tikzpicture}
        \memorylayout

        \only<2->{
          \stackframe[start=0,contents={bar.x = 1}]
        }
        \only<3->{
          \stackframe[start=1,contents={bar.y = 2}]
        }
        \only<5-7>{
          \stackframe[start=2,contents={b.x = 1}]
        }
        \only<8-10>{
          \stackframe[start=2,contents={b.x = 2}]
        }
        \only<6-9>{
          \stackframe[start=3,contents={b.y = 2}]
        }
      \end{tikzpicture}
    \end{columns}
  \end{center}
  \vskip2mm
  \begin{overprint}
    \onslide<1-3>
    \begin{center}
      Creation of bar on stack
    \end{center}

    \onslide<4-6>
    \begin{center}
      Calling {\tt foo}: {\tt bar} is passed by value \\
      $\Rightarrow$ a copy is made \\
      $\Rightarrow$ copy constructor is called
    \end{center}

    \onslide<7-8>
    \begin{center}
      Incrementing {\tt b.x} operates on copy
    \end{center}

    \onslide<9-11>
    \begin{center}
      Returning from {\tt foo}: all locals are cleaned up \\
      $\Rightarrow$ {\tt b}'s destructor is called \\
    \end{center}

    \onslide<12>
    \begin{center}
      {\tt bar} is still in its original state
    \end{center}
  \end{overprint}
\end{frame}
