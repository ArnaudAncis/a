\begin{frame}
  \frametitle{Return By Value}
  \code[frame=lines,width=.5\linewidth,font=\small]{return-by-value.cpp}
  \begin{itemize}
    \item {\tt foo} returns a {\tt Bar}
    \item This means it returns it \emph{by value}
    \item A copy must be made
    \item This copy is made using a copy constructor
    \item In reality, things are more complex
          \begin{itemize}
            \item Compiler can optimise: have {\tt foo} directly write to {\tt bar}
            \item Move constructors can be used under certain circumstances
          \end{itemize}
  \end{itemize}
\end{frame}

\begin{frame}
  \frametitle{Return By Value}
  \begin{center}
    \begin{columns}
      \column{4cm}
      \code[frame=lines,width=.95\linewidth,font=\small]{return-by-value.cpp}
      \column{4cm}
      \begin{tikzpicture}
        \memorylayout

        \only<2->{
          \stackframe[start=0,contents={bar.x = ?}]
        }
        \only<3->{
          \stackframe[start=1,contents={bar.y = ?}]
        }
        \only<6->{
          \stackframe[start=2,contents={b.x = 1}]
        }
        \only<7->{
          \stackframe[start=3,contents={b.y = 2}]
        }
      \end{tikzpicture}
    \end{columns}
  \end{center}
  \vskip2mm
  \begin{overprint}
    \onslide<1-3>
    \begin{center}
      Space for {\tt bar} is allocated on stack \\
      {\tt bar} is \emph{not} initialised yet \\
      We have to wait for {\tt foo()}'s result
    \end{center}

    \onslide<4>
    \begin{center}
      {\tt foo} gets called.
    \end{center}

    \onslide<5-7>
    \begin{center}
      {\tt b} gets created on the stack.
    \end{center}

    \onslide<8>
    \begin{center}
      What happens next depends on your compiler and its optimisation settings\dots
    \end{center}
  \end{overprint}
\end{frame}

\begin{frame}
  \begin{center} \Huge
    Possible Path A \\[4mm]
    The Path of Least Optimisation
  \end{center}
\end{frame}

\begin{frame}
  \frametitle{Return By Value (Path A)}
  \begin{center}
    \begin{columns}
      \column{4cm}
      \code[frame=lines,width=.95\linewidth,font=\small]{return-by-value.cpp}
      \column{4cm}
      \begin{tikzpicture}
        \memorylayout

        \only<1-6>{
          \stackframe[start=0,contents={bar.x = ?}]
        }
        \only<7->{
          \stackframe[start=0,contents={bar.x = 3}]
        }
        \only<1-7>{
          \stackframe[start=1,contents={bar.y = ?}]
        }
        \only<8-13>{
          \stackframe[start=1,contents={bar.y = 4}]
        }
        \only<1-12>{
          \stackframe[start=2,contents={b.x = 1}]
        }
        \only<1-11>{
          \stackframe[start=3,contents={b.y = 2}]
        }
        \only<2-10>{
          \stackframe[start=4,contents={temp.x = 2}]
        }
        \only<3-9>{
          \stackframe[start=5,contents={temp.y = 3}]
        }
      \end{tikzpicture}
    \end{columns}
  \end{center}
  \vskip2mm
  \begin{overprint}
    \onslide<1-3>
    \begin{center}
      Returning {\tt b} creates a copy of {\tt b} \\
      Let's call it {\tt temp} \\
    \end{center}

    \onslide<4>
    \begin{center}
      {\tt temp} is considered the object that is returned by {\tt foo}
    \end{center}

    \onslide<5>
    \begin{center}
      We can continue initialising {\tt bar}. \\
      If declaration and assignment occur on the same line (as is the case here),
      the copy constructor is called.
    \end{center}

    \onslide<6-8>
    \begin{center}
      The copy constructor is called with a reference to {\tt temp} as its argument
    \end{center}

    \onslide<9-13>
    \begin{center}
      {\tt temp} and {\tt b} are cleaned up \\
      {\tt \~{}Bar()} is called twice
    \end{center}
  \end{overprint}
\end{frame}

\begin{frame}
  \begin{center} \Huge
    Possible Path B \\[4mm]
    Return Value Optimisation (RVO)
  \end{center}
\end{frame}

\begin{frame}
  \frametitle{Return By Value (Path B)}
  \begin{center}
    \begin{columns}
      \column{4cm}
      \code[frame=lines,width=.95\linewidth,font=\small]{return-by-value.cpp}
      \column{4cm}
      \begin{tikzpicture}
        \memorylayout

        \only<1-2>{
          \stackframe[start=0,contents={bar.x = ?}]
        }
        \only<3->{
          \stackframe[start=0,contents={bar.x = 2}]
        }
        \only<1-3>{
          \stackframe[start=1,contents={bar.y = ?}]
        }
        \only<4->{
          \stackframe[start=1,contents={bar.y = 3}]
        }
        \only<1-6>{
          \stackframe[start=2,contents={b.x = 1}]
        }
        \only<1-5>{
          \stackframe[start=3,contents={b.y = 2}]
        }
      \end{tikzpicture}
    \end{columns}
  \end{center}
  \vskip2mm
  \begin{overprint}
    \onslide<1>
    \begin{center}
      The compiler is allowed to omit making a copy of {\tt b} prior to returning it \\
      {\tt b} is then considered the object that is returned
    \end{center}

    \onslide<2-4>
    \begin{center}
      The copy constructor initialising {\tt bar} is called with a reference to {\tt b} as argument
    \end{center}

    \onslide<5-7>
    \begin{center}
      {\tt b}'s destructor is called
    \end{center}
  \end{overprint}
\end{frame}

\begin{frame}
  \begin{center} \Huge
    Possible Path C \\[4mm]
    Full Optimisation
  \end{center}
\end{frame}

\begin{frame}
  \frametitle{Return By Value (Path C)}
  \begin{center}
    \begin{columns}
      \column{4cm}
      \code[frame=lines,width=.95\linewidth,font=\small]{return-by-value.cpp}
      \column{4cm}
      \begin{tikzpicture}
        \memorylayout

        \only<1-3>{
          \stackframe[start=0,contents={bar.x = ?},id=bar]
        }
        \only<4->{
          \stackframe[start=0,contents={bar.x = 1},id=bar]
        }
        \only<1-4>{
          \stackframe[start=1,contents={bar.y = ?}]
        }
        \only<5->{
          \stackframe[start=1,contents={bar.y = 2}]
        }
        \only<1-6>{
          \stackframe[start=2,contents={b},id=b]
          \draw[-latex] (b.east) to [bend right=30] (bar.east);
        }
      \end{tikzpicture}
    \end{columns}
  \end{center}
  \vskip2mm
  \begin{overprint}
    \onslide<1>
    \begin{center}
      It is not necessary for {\tt foo} to have its own object {\tt b}.
      Instead, it can operate directly on {\tt bar}.
    \end{center}

    \onslide<2>
    \begin{center}
      Every write that {\tt foo} makes to {\tt b}
      is then in reality a write to {\tt bar}
    \end{center}

    \onslide<3-5>
    \begin{center}
      When {\tt foo} initialised {\tt b} using the default constructor,
      it is in reality initialising {\tt bar}
    \end{center}

    \onslide<6-7>
    \begin{center}
      When {\tt foo} ends, {\tt b} is cleaned up.
      Since {\tt b} was never constructed as a separate object, {\tt b} cannot be destructed either.
      In other words, {\tt b} just disappears. {\tt b} can be seen as a reference to {\tt bar}.
    \end{center}
  \end{overprint}
\end{frame}

\begin{frame}
  \begin{center} \Huge
    Possible Path D \\[4mm]
    Move Constructor
  \end{center}
\end{frame}

\begin{frame}
  \frametitle{Return By Value (Path D)}
  \begin{itemize}
    \item If a move constructor is defined, {\tt b}
          may be \emph{moved} to {\tt bar}
    \item Steps:
          \begin{enumerate}
            \item {\tt b} gets created with the default constructor
            \item Upon returning, {\tt bar} gets initialised with its move constructor.
                  It ``steals'' {\tt b}'s data
            \item {\tt b} is destructed
          \end{enumerate}
    \item We don't discuss move constructors in detail for this course
  \end{itemize}
\end{frame}

\begin{frame}
  \frametitle{Return By Value: Summary}
  \begin{itemize}
    \item It is impossible to predict what exactly will happen
    \item Four possibilities
          \begin{itemize}
            \item Two copy constructors called
            \item One copy constructor called
            \item Zero constructors called
            \item One move constructors called
          \end{itemize}
    \item It is therefore important to define your constructors in such a way
          that it does not matter how often they get called
    \item A copy constructor should only copy and do nothing more
    \item Then it does not matter if you get an object, a copy of an object, or a copy of a copy of an object
    \item {\tt Bar}'s implementation clearly does not satisfy this requirement
  \end{itemize}
\end{frame}

%%% Local Variables:
%%% mode: latex
%%% TeX-master: "technical-details"
%%% End:
