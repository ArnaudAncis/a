\documentclass{../ucll-slides}
\usepackage{../pvm}
\usetikzlibrary{shadows,shapes.multipart}

\title{Classes: Constructors}
\author{Fr\'ed\'eric Vogels}

\newcommand{\highlightbox}[2][]{
  \draw[opacity=.75,ultra thick,red,#1] ($ (#2.south west) + (-.1,-.1) $) rectangle ($ (#2.north east) + (.1,.1) $)
}


\begin{document}

% \begin{frame}
%   \titlepage
% \end{frame}

% \begin{frame}
%   \frametitle{Special Constructors}
%   \begin{itemize}
%     \item \cpp\ has a few ``special'' constructors
%     \item They get called in specific circumstances
%     \item New stuff in \cpp11!
%   \end{itemize}
% \end{frame}

% \begin{frame}
%   \frametitle{Default Constructor}
%   \begin{itemize}
%     \item Constructor with zero parameters
%   \end{itemize}
%   \code[frame=lines,width=.6\linewidth]{default-constructor-example.cpp}
% \end{frame}

% \begin{frame}
%   \frametitle{Default Constructor: When Is It Called?}
%   \begin{itemize}
%     \item When creating an object without giving arguments
%     \item Array creation
%   \end{itemize}
%   \vskip5mm
%   \structure{Examples}
%   \code[frame=lines,width=.9\linewidth]{default-constructor-calls.cpp}
% \end{frame}

% \begin{frame}
%   \frametitle{Default Constructor: Automatic Generation}
%   \begin{itemize}
%     \item Generated automatically if you do not define any constructors yourself
%           \begin{itemize}
%             \item Calls base constructors
%             \item Calls default constructors on member variables
%             \item Does \emph{not} initialise member variables to zero! \\ Default constructor for {\tt int} does not set value to {\tt 0}
%           \end{itemize}
%     \item Generation \link{http://en.cppreference.com/w/cpp/language/default_constructor}{rules} are quite complex
%     \item It is easier to be explicit about default constructors
%   \end{itemize}
% \end{frame}

% \begin{frame}
%   \frametitle{Default Constructor}
%   \begin{itemize}
%     \item Forcing no default constructor
%           \vskip1mm
%           \code[frame=lines,font size=\small]{no-default-constructor.cpp}
%     \item Force default constructor generation
%           \vskip1mm
%           \code[frame=lines,font size=\small]{force-default-constructor.cpp}
%   \end{itemize}
% \end{frame}

% \begin{frame}
%   \frametitle{Default Constructor}
%   \begin{itemize}
%     \item Custom default constructor
%           \vskip1mm
%           \code[frame=lines,font size=\small]{custom-default-constructor.cpp}
%   \end{itemize}
% \end{frame}

% \begin{frame}
%   \frametitle{Default Constructor: Quiz}
%   \code[frame=lines]{default-constructor.cpp}
%   \visible<2->{
%     \begin{center}
%       Does not compile! \\ Declaration of {\tt foo} misinterpreted as function declaration \\
%       Known as \link{https://en.wikipedia.org/wiki/Most_vexing_parse}{Most Vexing Parse}
%     \end{center}
%   }
% \end{frame}

% \begin{frame}
%   \frametitle{Default Constructor: Quiz}
%   \code[frame=lines]{default-constructor2.cpp}
% \end{frame}

% \begin{frame}
%   \frametitle{Default Constructor: Quiz}
%   \code[frame=lines]{default-constructor3.cpp}
%   \visible<2->{
%     \begin{center}
%       Does not compile! Parentheses are necessary here!
%     \end{center}
%   }
% \end{frame}

% \begin{frame}
%   \frametitle{Copy Constructor}
%   \begin{itemize}
%     \item Constructor taking {\tt const T\&}
%     \item Should make (deep) copy of given object
%     \item Called whenever a copy needs to be made
%   \end{itemize}
%   \vskip5mm
%   \code[frame=lines]{copy-constructor.cpp}
% \end{frame}

% \begin{frame}
%   \frametitle{Copy Constructor: Automatic Generation}
%   \begin{itemize}
%     \item Automatically generated, similar \link{http://en.cppreference.com/w/cpp/language/copy_constructor}{rules} to default constructor
%     \item Can be explicitly generated (using {\tt = default})
%     \item Can be explicitly omitted (using {\tt = delete})
%   \end{itemize}
%   \code[frame=lines,font size=\small]{implicit-copy-constructor.cpp}
% \end{frame}

% \begin{frame}
%   \frametitle{Copy Constructor: When Is It Called?}
%   \begin{itemize}
%     \item When copy of object is needed
%     \item In some cases, \cpp\ standard allows \link{https://en.wikipedia.org/wiki/Return_value_optimization}{optimisations}
%           that change the behaviour of the program, so maybe copy constructor does not even get called
%     \item It is therefore important not to do exotic things in the copy constructor besides copying the object
%   \end{itemize}
% \end{frame}

% \begin{frame}
%   \frametitle{Copy Constructor: When Is It Called?}
%   \code[frame=lines]{copy-constructor1.cpp}
% \end{frame}

% \begin{frame}
%   \frametitle{Copy Constructor: When Is It Called?}
%   \code[frame=lines]{copy-constructor2.cpp}
% \end{frame}

% \begin{frame}
%   \frametitle{Copy Constructor: When Is It Called?}
%   \code[frame=lines]{copy-constructor3.cpp}
% \end{frame}

% \begin{frame}
%   \frametitle{Copy Constructor: When Is It Called?}
%   \code[frame=lines]{copy-constructor4.cpp}
% \end{frame}

% \begin{frame}
%   \frametitle{Copy Constructor: When Is It Called?}
%   \code[frame=lines]{copy-constructor5.cpp}
% \end{frame}

% \begin{frame}
%   \frametitle{Copy Constructor: When Is It Called?}
%   \code[frame=lines]{copy-constructor6.cpp}
% \end{frame}

% \begin{frame}
%   \frametitle{Move Constructor}
%   \begin{itemize}
%     \item Introduced in \cpp11
%     \item Optimisation
%     \item Called when object is ``moved''
%   \end{itemize}
% \end{frame}

% \begin{frame}
%   \frametitle{Example}
%   \code[frame=lines]{move-example.cpp}
%   \begin{itemize}
%     \item When {\tt foo} returns, {\tt ns} must be copied to {\tt result}
%     \item Since {\tt foo} returns, {\tt ns} must be destroyed too
%     \item So, we're making a copy only to destroy the original
%     \item Why copy at all?
%   \end{itemize}
% \end{frame}

% \begin{frame}
%   \frametitle{Move Constructor}
%   \begin{itemize}
%     \item Move constructor can be used in such circumstances
%     \item One object ``takes over'' all data from the original
%     \item The original object is ``emptied''
%     \item Emptying original is important, as its destructor will still be called
%   \end{itemize}
%   \code[frame=lines,width=.6\linewidth]{move-constructor-syntax.cpp}
% \end{frame}

% \begin{frame}
%   \frametitle{Example}
%   \code[frame=lines,font size=\small]{move-constructor.cpp}
% \end{frame}

% \begin{frame}
%   \frametitle{Visualisation: Copy Constructor}
%   \begin{center}
%     \begin{tikzpicture}[object/.style={draw,minimum width=3cm,minimum height=.75cm}]
%       \visible<1-20>{
%         \node[object] (v1) at (0,0) {\tt vector<int>};
%       }

%       \visible<1-19>{
%         \draw[xstep=.25cm,ystep=.25cm,xshift=-0.125cm] (0,-1) grid (0.25,-5);
%         \draw[-latex] (v1.south) -- (0,-1);
%         \foreach[count=\i] \c in {a,b,c,d,e,f,g,h,i,j,k,l,m,n,o,p} {
%           \tikzmath{
%             real \j;
%             \j = -1 - 0.25 * \i + 0.125;
%           }

%           \node[font=\tiny] at (0,\j) {\c};
%         }
%       }

%       \visible<2-21>{
%         \node[object] (v2) at (5,0) {\tt vector<int>};
%       }

%       \visible<3->{
%         \draw[xstep=.25cm,ystep=.25cm,xshift=4.875cm] (0,-1) grid (0.25,-5);
%         \draw[-latex] (v2.south) -- (5,-1);
%       }

%       \foreach[count=\i] \c in {a,b,c,d,e,f,g,h,i,j,k,l,m,n,o,p} {
%         \tikzmath{
%           int \slideindex;
%           real \j;
%           \slideindex = int(\i + 3);
%           \j = -1 - 0.25 * \i + 0.125;
%         }

%         \visible<\slideindex>{
%           \draw[-latex] (0,\j) -- (5,\j);
%         }

%         \visible<\slideindex->{
%           \node[font=\tiny] at (5,\j) {\c};
%         }
%       }
%     \end{tikzpicture}
%   \end{center}
% \end{frame}

% \begin{frame}
%   \frametitle{Visualisation: Move Constructor}
%   \begin{center}
%     \begin{tikzpicture}[object/.style={draw,minimum width=3cm,minimum height=.75cm}]
%       \visible<1-3>{
%         \node[object] (v1) at (0,0) {\tt vector<int>};
%       }

%       \draw[xstep=.25cm,ystep=.25cm,xshift=-.125cm] (0,-1) grid (0.25,-5);
%       \foreach[count=\i] \c in {a,b,c,d,e,f,g,h,i,j,k,l,m,n,o,p} {
%         \tikzmath{
%           real \j;
%           \j = -1 - 0.25 * \i + 0.125;
%         }

%         \node[font=\tiny] at (0,\j) {\c};
%       }

%       \visible<1-2>{
%         \draw[-latex] (v1.south) -- (0,-1);
%       }

%       \visible<2-4>{
%         \node[object] (v2) at (5,0) {\tt vector<int>};
%       }

%       \visible<3>{
%         \draw[-latex] (v2.south) -- (0,-1) node[midway,below,sloped] {yoink!};
%       }

%       \visible<4>{
%         \draw[-latex] (v2.south) -- (0,-1);
%       }
%     \end{tikzpicture}
%   \end{center}
% \end{frame}

\begin{frame}
  \frametitle{Unary Constructors}
  \begin{itemize}
    \item Unary constructors: constructors with single argument
    \item Act as implicit casts
  \end{itemize}
  \vskip5mm
  \begin{overprint}
    \onslide<1>
    \code[frame=lines,font size=\small]{implicit1.cpp}
    \onslide<2>
    \code[frame=lines,font size=\small]{implicit2.cpp}
  \end{overprint}
\end{frame}

\begin{frame}
  \frametitle{Preventing Implicit Casts}
  \begin{itemize}
    \item Generally, you don't want implicit casts
    \item Add {\tt explicit} before unary constructor
  \end{itemize}
  \code[frame=lines,font size=\small]{explicit.cpp}
\end{frame}

\end{document}