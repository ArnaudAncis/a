\documentclass{../ucll-slides}
\usepackage{../pvm}
\usetikzlibrary{shadows,shapes.multipart}

\title{Classes: Constructors}
\author{Fr\'ed\'eric Vogels}

\newcommand{\highlightbox}[2][]{
  \draw[opacity=.75,ultra thick,red,#1] ($ (#2.south west) + (-.1,-.1) $) rectangle ($ (#2.north east) + (.1,.1) $)
}


\begin{document}

\begin{frame}
  \titlepage
\end{frame}

\begin{frame}
  \frametitle{Default Constructor}
  \begin{itemize}
    \item Constructor with zero parameters
    \item Generated automatically if you do not define any constructors yourself
          \begin{itemize}
            \item Calls base constructors
            \item Calls default constructors on member variables
            \item Does \emph{not} initialise fields to zero!
          \end{itemize}
    \item Generation \link{http://en.cppreference.com/w/cpp/language/default_constructor}{rules} are quite complex
    \item It is easier to be explicit about default constructors
  \end{itemize}
\end{frame}

\begin{frame}
  \frametitle{Default Constructor}
  \begin{itemize}
    \item Forcing no default constructor
          \vskip1mm
          \code[frame=lines,font size=\small]{no-default-constructor.cpp}
    \item Force default constructor generation
          \vskip1mm
          \code[frame=lines,font size=\small]{force-default-constructor.cpp}
  \end{itemize}
\end{frame}

\begin{frame}
  \frametitle{Default Constructor}
  \begin{itemize}
    \item Custom default constructor
          \vskip1mm
          \code[frame=lines,font size=\small]{custom-default-constructor.cpp}
  \end{itemize}
\end{frame}

\begin{frame}
  \frametitle{Default Constructor: Quiz}
  \code[frame=lines]{default-constructor.cpp}
  \visible<2->{
    \begin{center}
      Does not compile! \\ Declaration of {\tt foo} misinterpreted as function declaration
    \end{center}
  }
\end{frame}

\begin{frame}
  \frametitle{Default Constructor: Quiz}
  \code[frame=lines]{default-constructor2.cpp}
\end{frame}

\begin{frame}
  \frametitle{Default Constructor: Quiz}
  \code[frame=lines]{default-constructor3.cpp}
  \visible<2->{
    \begin{center}
      Does not compile! Parentheses are necessary here!
    \end{center}
  }
\end{frame}

\begin{frame}
  \frametitle{Copy Constructor}
  \begin{itemize}
    \item Constructor taking {\tt const T\&}
    \item Should make copy of given object
    \item Automatically generated, similar \link{http://en.cppreference.com/w/cpp/language/copy_constructor}{rules} to default constructor
    \item Can be explicitly generated (using {\tt = default})
    \item Can be explicitly omitted (using {\tt = delete})
    \item Called whenever a copy needs to be made
  \end{itemize}
  \vskip5mm
  \code[frame=lines]{copy-constructor.cpp}
\end{frame}

\begin{frame}
  \frametitle{Working Example}
  \code[frame=lines,font size=\small]{oiltank.cpp}
\end{frame}

\begin{frame}
  \frametitle{Quiz}
  \structure{Does the copy constructor get called?}
  \vskip5mm
  \code[frame=lines]{copy-quiz1.cpp}
  \visible<2>{
    \begin{center}
      Yes!
    \end{center}
  }
\end{frame}

\begin{frame}
  \frametitle{Quiz}
  \structure{Does the copy constructor get called?}
  \vskip5mm
  \code[frame=lines]{copy-quiz2.cpp}
  \visible<2>{
    \begin{center}
      Yes!
    \end{center}
  }
\end{frame}

\begin{frame}
  \frametitle{Quiz}
  \structure{Does the copy constructor get called?}
  \vskip5mm
  \code[frame=lines]{copy-quiz3.cpp}
  \visible<2>{
    \begin{center}
      No! \\
      This calls {\tt operator =(const OilTank\&)} (see later)
    \end{center}
  }
\end{frame}

\begin{frame}
  \frametitle{Quiz}
  \structure{Does the copy constructor get called?}
  \vskip5mm
  \code[frame=lines]{copy-quiz4.cpp}
  \visible<2>{
    \begin{center}
      No!
    \end{center}
  }
\end{frame}

\begin{frame}
  \frametitle{Quiz}
  \structure{Does the copy constructor get called?}
  \vskip5mm
  \code[frame=lines]{copy-quiz5.cpp}
  \visible<2>{
    \begin{center}
      Yes!
    \end{center}
  }
\end{frame}

\begin{frame}
  \frametitle{Quiz}
  \structure{Does the copy constructor get called?}
  \vskip5mm
  \code[frame=lines]{copy-quiz6.cpp}
  \visible<2>{
    \begin{center}
      No!
    \end{center}
  }
\end{frame}

\begin{frame}
  \frametitle{Quiz}
  \structure{Does the copy constructor get called?}
  \vskip5mm
  \code[frame=lines]{copy-quiz7.cpp}
  \visible<2>{
    \begin{center}
      No!
    \end{center}
  }
\end{frame}

\begin{frame}
  \frametitle{Quiz}
  \structure{Does the copy constructor get called?}
  \vskip5mm
  \code[frame=lines]{copy-quiz8.cpp}
  \visible<2>{
    \begin{center}
      No!
    \end{center}
  }
\end{frame}

\begin{frame}
  \frametitle{Quiz}
  \structure{Does the copy constructor get called?}
  \vskip5mm
  \code[frame=lines]{copy-quiz9.cpp}
  \visible<2>{
    \begin{center}
      Yes!
    \end{center}
  }
\end{frame}

\begin{frame}
  \frametitle{Quiz}
  \structure{Does the copy constructor get called?}
  \vskip5mm
  \code[frame=lines]{copy-quiz10.cpp}
  \visible<2>{
    \begin{center}
      Yes!
    \end{center}
  }
\end{frame}

\begin{frame}
  \frametitle{Rules}
  \begin{itemize}
    \item Pre \cpp11: copy constructor called when copy needed
      \begin{itemize}
        \item Parameter pass-by-value
        \item Returned by value
      \end{itemize}
    \item \cpp11 introduced \emph{move constructor}
      \begin{itemize}
        \item Optimisation
        \item Called when object needs to be copied, but original is immediately discarded
      \end{itemize}
  \end{itemize}
\end{frame}

% TODO: Rationale

\begin{frame}
  \frametitle{Examples}
  \begin{overprint}
    \onslide<1>
    \code[frame=lines]{move-constructor1.cpp}

    \onslide<2>
    \code[frame=lines]{move-constructor2.cpp}

    \onslide<3>
    \code[frame=lines]{move-constructor3.cpp}

    \onslide<4>
    \code[frame=lines]{move-constructor4.cpp}

    \onslide<5>
    \code[frame=lines]{move-constructor5.cpp}

    \onslide<6>
    \code[frame=lines]{move-constructor6.cpp}
  \end{overprint}
  \vskip5mm
  \begin{overprint}
    \onslide<1>
    \begin{itemize}
      \item Copy constructor is called
      \item Because {\tt t1} still exists afterwards
    \end{itemize}

    \onslide<2>
    \begin{itemize}
      \item Move constructor is called
      \item Because {\tt OilTank(0, 100)} is temporary
    \end{itemize}

    \onslide<3>
    \begin{itemize}
      \item Move constructor is called
      \item Because {\tt OilTank(0, 100)} is temporary
    \end{itemize}

    \onslide<4>
    \begin{itemize}
      \item Move constructor is called
      \item Because {\tt t} goes out of scope
    \end{itemize}

    \onslide<5>
    \begin{itemize}
      \item Copy constructor is called
      \item Because {\tt t} goes on existing after function is done
    \end{itemize}

    \onslide<6>
    \begin{itemize}
      \item Move constructor is called
      \item It does not matter that result of {\tt foo()} is not stored
      \item It's as if the result is stored in a hidden temporary variable
      \item Move, although useless here, is still performed because it might have side effects (e.g. print something)
    \end{itemize}
  \end{overprint}
\end{frame}


\end{document}