\documentclass{../ucll-slides}
\usepackage{../pvm}
\usepackage{siunitx}
\usepackage{ifthen}

\usetikzlibrary{positioning}

\makeatletter
\def\light@caption@on{on}
\def\light@caption@off{off}
\def\light@caption@neutral{}
\makeatother

\newcommand{\hex}[1]{\texttt{\bfseries #1}}


\title{Data Compression: LZ77}
\author{Fr\'ed\'eric Vogels}

\begin{document}

\begin{frame}
  \titlepage
\end{frame}

% \begin{frame}
%   \frametitle{Disclaimer}
%   \begin{center}
%     \Large
%     Most examples assume we deal with text,
%     but the principles apply to bytes in general
%   \end{center}
% \end{frame}

% \begin{frame}
%   \frametitle{Compression Algorithms}
%   \begin{itemize}
%     \item GQU
%           \begin{itemize}
%             \item Example algorithm
%             \item Assumes pairs of bytes occur often
%           \end{itemize}
%     \item RLE
%           \begin{itemize}
%             \item Assumes long runs of the same value occur often
%             \item TGA, PCX, fax machines
%           \end{itemize}
%     \item Huffman
%           \begin{itemize}
%             \item Assumes certain bytes occur more often
%             \item Widely used (MP3, JPEG, zip, web, \dots)
%           \end{itemize}
%     \item<2-> LZ77/LZ78/LZSS/LZMA/\dots
%           \begin{itemize}
%             \item Assumes same series of bytes occur repeatedly
%             \item Widely used (zip, 7z, rar, \dots)
%           \end{itemize}
%   \end{itemize}
% \end{frame}

% {
%   \newcommand{\drawbox}[1]{\draw (#1.south west) rectangle (#1.north east);}
%   \begin{frame}
%     \frametitle{Example Input}
%     \code[language={},font size=\small,width=\linewidth,frame=none]{singer.txt}
%     \begin{tikzpicture}[overlay,remember picture,ref/.style={-latex,thick}]
%       \visible<2->{
%         \drawbox{inger1}
%         \drawbox{inger2}
%         \draw[ref] (inger2.east) to[bend right=30] (inger1.east);
%       }

%       \visible<3->{
%         \drawbox{hoort1}
%         \drawbox{hoort2}
%         \draw[ref] (hoort2.east) to[bend right=30] (hoort1.east);
%       }

%       \visible<4->{
%         \draw (singer1.north west) rectangle (inger2.south east);
%         \drawbox{singer2}
%         \draw[ref] (singer2.north) to[bend right=30] (inger2.east);
%       }

%       \visible<5->{
%         \drawbox{naaimasjien1}
%         \drawbox{naaimasjien2}
%         \draw[ref] (naaimasjien2.north) to[bend right=30] (naaimasjien1.east);
%       }

%       \visible<6->{
%         \drawbox{wat1}
%         \drawbox{wat2}
%         \draw[ref] (wat2.east) to[bend right=30] (wat1.east);
%       }

%       \visible<7->{
%         \drawbox{jespers1}
%         \drawbox{jespers2}
%         \draw[ref] (jespers2.east) to[bend right=30] (jespers1.south);
%       }

%       \visible<8->{
%         \drawbox{singer3}
%         \drawbox{singer2}
%         \draw[ref] (singer3.north east) to[bend right=10] (singer2.south);
%       }

%       \visible<9->{
%         \drawbox{naaimasjien3}
%         \drawbox{naaimasjien2}
%         \draw[ref] (naaimasjien3.north east) to[bend right=10] (naaimasjien2.south);
%       }
%     \end{tikzpicture}
%   \end{frame}
% }

% \begin{frame}
%   \frametitle{Observations}
%   \begin{itemize}
%     \item Same words are repeated often
%     \item Happens often in certain data files
%           \begin{itemize}
%             \item Source code (keywords, identifiers, etc.)
%             \item Text files (word, pdf, etc.)
%             \item XML
%             \item \dots
%           \end{itemize}
%     \item More generally: certain series of bytes occur repeatedly
%     \item We can use this to our advantage
%   \end{itemize}
% \end{frame}

% \begin{frame}
%   \frametitle{Basic Principle of LZ77}
%   \begin{itemize}
%     \item If same word has appeared before\dots
%     \item \dots don't repeat it
%     \item \dots but refer to the previous occurence
%   \end{itemize}
% \end{frame}

% {
%   \newcommand{\chunk}[3]{
%     \draw[|-|] ($ (#1.south west) + (0,-0.1) $) -- ($ (#1.south east) + (0,-0.1) $);
%     \draw[|->] ($ (#2.south west) + (0,-0.1) $) -- ($ (#2.south east) + (0,-0.1) $) node[midway,below,font=\tiny] {#3};
%     \draw[-latex] ($ (#1.south) + (0,-0.1) $) -- ++(0,-0.5) -| ($ (#2.south west) + (0,-0.2) $);
%   }
%   \newcommand{\multichunk}[5]{
%     \draw[|-|] ($ (#1.south west) + (0,-0.1) $) -- ($ (#2.south east) + (0,-0.1) $);
%     \draw[|->] ($ (#3.south west) + (0,-0.1) $) -- ($ (#4.south east) + (0,-0.1) $) node[midway,below,font=\tiny] {#5};
%     \draw[-latex] ($ (#1.south) ! 0.5 ! (#2.south) + (0,-0.1) $) -- ++(0,-0.5) -| ($ (#3.south west) + (0,-0.2) $);
%   }
%   \begin{frame}
%     \frametitle{Conceptual Example}
%     \structure{Data to be compressed}
%     \begin{center} \tt
%       \NODE{\alert<2>{mux}}{mux1}\NODE{\alert<3>{mux}}{mux2}\NODE{\alert<4>{bar}}{bar1}\NODE{\alert<5>{bar}}{bar2}\NODE{\alert<6>{muxbarbar}}{muxbarbar}\NODE{\alert<7>{muxmuxbarbarmuxbarbar}}{last}
%     \end{center}
%     \begin{tikzpicture}[overlay,remember picture]
%       \visible<3>{
%         \chunk{mux2}{mux1}{3}
%       }

%       \visible<5>{
%         \chunk{bar2}{bar1}{3}
%       }

%       \visible<6>{
%         \multichunk{muxbarbar}{muxbarbar}{mux2}{bar2}{6}
%       }

%       \visible<7>{
%         \multichunk{last}{last}{mux1}{muxbarbar}{21}
%       }
%     \end{tikzpicture}
%     \vskip5mm
%     \structure{Compressed form}
%     \begin{enumerate}
%       \item<2-> {\tt mux}
%       \item<3-> Repeat 3 letters starting from first letter
%       \item<4-> {\tt bar}
%       \item<5-> Repeat 3 letters starting from 6th letter
%       \item<6-> Repeat 6 letters starting from 3rd letter
%       \item<7-> Repeat 21 letters starting from first letter
%     \end{enumerate}
%   \end{frame}
% }

% \begin{frame}
%   \frametitle{Resolving Technical Difficulties}
%   \begin{itemize}
%     \item Conceptually, LZ works by referring to past data
%     \item We still need to find a way to encode this into bytes
%     \item We need to make sure a decompressor can unambiguously reconstruct the original data
%   \end{itemize}
% \end{frame}

% \begin{frame}
%   \frametitle{First Attempt}
%   \begin{itemize}
%     \item A back reference consists of two components
%           \begin{itemize}
%             \item Start location of data to be repeated
%             \item How many bytes to repeat
%           \end{itemize}
%     \item Let's use a single byte for each
%   \end{itemize}

%   \begin{center}
%     \begin{tabular}{l@{$\;\rightarrow\;$}l}
%       {\tt mux} & \tt \hex{6D} \hex{75} \hex{78} \\
%       Repeat 3 letters starting from first letter & \tt \hex{00} \hex{03} \\
%       {\tt bar} & \tt \hex{62} \hex{61} \hex{72} \\
%       Repeat 3 letters starting from 6th letter & \tt \hex{05} \hex{03} \\
%       Repeat 6 letters starting from 3rd letter & \tt \hex{02} \hex{06} \\
%       Repeat 21 letters starting from first letter & \tt \hex{00} \hex{15} \\
%     \end{tabular}
%   \end{center}
% \end{frame}

% \begin{frame}
%   \frametitle{Ambiguity Problem}
%   \begin{center}
%     \begin{tikzpicture}
%       \node[font=\tt,draw] (original) at (0,0) {\hex{AA} \hex{00} \hex{01}};
%       \node[font=\tt,anchor=east,draw] (decompression 1) at (-2,-2) {\hex{AA} \hex{00} \hex{01}};
%       \node[font=\tt,anchor=west,draw] (decompression 2) at (2,-2) {\hex{AA} \hex{AA}};

%       \draw[-latex] (original) -- (decompression 1.north east) node[midway,above,sloped,font=\tiny] {interpretation 1};
%       \draw[-latex] (original) -- (decompression 2.north west) node[midway,above,sloped,font=\tiny] {interpretation 2};
%     \end{tikzpicture}
%   \end{center}

%   \begin{itemize}
%     \item How to differentiate between
%           \begin{itemize}
%             \item A literal byte
%             \item A reference to previous data
%           \end{itemize}
%     \item Interpretation 1: {\tt 00 01} is data
%     \item Interpretation 2: {\tt 00 01} is a reference to past data \\ (start at index 0, length 1)
%   \end{itemize}
% \end{frame}

% \begin{frame}
%   \frametitle{Ambiguity Solution}
%   \begin{itemize}
%     \item Many solutions possible
%     \item LZ77 imposes uniformity
%     \item LZ77 does not discern between literal bytes and back reference bytes
%     \item In LZ77, everything is a back reference
%     \item This of course leads to a new problem
%   \end{itemize}
% \end{frame}

% \begin{frame}
%   \frametitle{The Case Of The Missing Literal Bytes}
%   \structure{Data to be compressed}
%   \begin{center}
%     \tt muxmuxbarbarmuxbarbarmuxmuxbarbarmuxbarbar
%   \end{center}
%   \vskip5mm
%   \structure{Our ordeal}
%   \begin{itemize}
%     \item LZ77 allows only back references
%     \item How to start compressing above data?
%     \item {\tt m} has never been encountered before
%     \item So no back reference possible
%     \item We're stuck!
%   \end{itemize}
% \end{frame}

% \begin{frame}
%   \frametitle{LZ77's Approach}
%   \begin{itemize}
%     \item LZ77 sees data as series of triplets
%     \item<2-> Location of repeated data
%               \begin{itemize}
%                 \item Expressed in ``how many bytes ago''
%                 \item Why? \cake
%               \end{itemize}
%     \item<3-> Length of repeated data
%     \item<4-> Single extra datum
%   \end{itemize}
%   \begin{center}
%     \begin{tikzpicture}[byte/.style={minimum width=0.75cm,minimum height=1cm,draw}]
%       \node[byte] (first) at (0,0) {};
%       \node[byte] (second) at (1,0) {};
%       \node[byte] (third) at (2,0) {};

%       \draw[|-|] ($ (first.south west) + (0,-0.5) $) -- ($ (third.south east) + (0,-0.5) $) node[midway,below,font=\small] {triplet};

%       \visible<2->{
%         \node[anchor=west] (loc) at (3,2) {Location};
%         \draw[-latex] (first.north) |- (loc.west);
%       }

%       \visible<3->{
%         \node[anchor=north west] (len) at (loc.south west) {Length};
%         \draw[-latex] (second.north) |- (len.west);
%       }

%       \visible<4->{
%         \node[anchor=north west] (slb) at (len.south west) {Datum};
%         \draw[-latex] (third.north) |- (slb.west);
%       }
%     \end{tikzpicture}
%   \end{center}
% \end{frame}

\begin{frame}
  \frametitle{Practical Example}
  \structure{Data to be compressed}
  \begin{center}
    \tt \alert<1-2>{m}\alert<3-4>{u}\alert<5-6>{x}{\only<7-8>{\color{green}}mux}\alert<7-8>{b}\alert<9-10>{a}\alert<11-12>{r}{\only<13-14>{\color{green}}bar}\alert<13-14>{m}%
    {\only<15-16>{\color{green}}uxbarbarm}\alert<15-16>{u}{\only<17-18>{\color{green}}xmuxbarbarmuxbarbar}\alert<17-18>{b}{\only<19-20>{\color{green}}armu}\alert<19-20>{x}
  \end{center}
  \structure{Explanation}
  \begin{overprint}
    \onslide<1-2>
    \begin{itemize}
      \item First datum: {\tt m} (hex \hex{6D})
      \item Has not appeared before
      \item We use 0 as position and length to indicate this
      \item Extra datum: \hex{6D}
    \end{itemize}

    \onslide<3-4>
    \begin{itemize}
      \item Second datum: {\tt u} (hex \hex{75})
      \item Has not appeared before
      \item We use 0 as position and length to indicate this
      \item Extra datum: \hex{75}
    \end{itemize}

    \onslide<5-6>
    \begin{itemize}
      \item Third datum: {\tt x} (hex \hex{78})
      \item Has not appeared before
      \item We use 0 as position and length to indicate this
      \item Extra datum: \hex{78}
    \end{itemize}

    \onslide<7-8>
    \begin{itemize}
      \item {\tt m} encountered before
      \item We take longest possible match
      \item {\tt\color{green} mux} is longest match
      \item Location = 3 bytes ago, length = 3
      \item Obligatory extra datum: {\tt b} (hex \hex{62})
    \end{itemize}

    \onslide<9-10>
    \begin{itemize}
      \item {\tt a} is new
      \item Location: 0, length = 0
      \item Extra datum: {\tt a} (hex \hex{61})
    \end{itemize}

    \onslide<11-12>
    \begin{itemize}
      \item {\tt r} is new
      \item Location: 0, length = 0
      \item Extra datum: {\tt r} (hex \hex{72})
    \end{itemize}

    \onslide<13-14>
    \begin{itemize}
      \item {\tt bar} looks familiar
      \item Location: 3 bytes ago, length = 3
      \item Extra datum: {\tt m} (hex \hex{6D})
    \end{itemize}

    \onslide<15-16>
    \begin{itemize}
      \item Longest match: {\tt uxbarbarm}
      \item Location: 9 bytes ago, length = 9
      \item Extra datum: {\tt u} (hex \hex{75})
    \end{itemize}

    \onslide<17-18>
    \begin{itemize}
      \item Longest match: {\tt xmuxbarbarmuxbarbar}
      \item Location: 21 bytes ago, length = 19
      \item 21 in hex: \hex{15}, 19 in hex: \hex{13}
      \item Extra datum: {\tt b} (hex \hex{62})
    \end{itemize}

    \onslide<19-20>
    \begin{itemize}
      \item Longest match: {\tt armux}
      \item Oops, no extra datum left
      \item Longest match that leaves extra datum: {\tt armu}
      \item Location: 12 bytes ago, length = 4
      \item Extra datum: {\tt x} (hex \hex{78})
    \end{itemize}

  \end{overprint}
  \vskip4mm
  \structure{Compressed form}
  \begin{center} \tt
    \parbox{9cm}{
      \visible<2->{\alert<2>{\hex{00} \hex{00} \hex{6D}}}
      \visible<4->{\alert<4>{\hex{00} \hex{00} \hex{75}}}
      \visible<6->{\alert<6>{\hex{00} \hex{00} \hex{78}}}
      \visible<8->{\alert<8>{\hex{03} \hex{03} \hex{62}}}
      \visible<10->{\alert<10>{\hex{00} \hex{00} \hex{61}}} \\
      \visible<12->{\alert<12>{\hex{00} \hex{00} \hex{72}}}
      \visible<14->{\alert<14>{\hex{03} \hex{03} \hex{6D}}}
      \visible<16->{\alert<16>{\hex{09} \hex{09} \hex{75}}}
      \visible<18->{\alert<18>{\hex{15} \hex{13} \hex{62}}}
      \visible<20->{\alert<20>{\hex{0C} \hex{04} \hex{78}}}
    }
  \end{center}
\end{frame}



\end{document}