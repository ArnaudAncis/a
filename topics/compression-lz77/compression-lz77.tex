\documentclass{../ucll-slides}
\usepackage{../pvm}
\usepackage{siunitx}
\usepackage{ifthen}

\usetikzlibrary{positioning}

\makeatletter
\def\light@caption@on{on}
\def\light@caption@off{off}
\def\light@caption@neutral{}
\makeatother

\tikzset{
  light/.pic={\draw node[circle,#1,minimum size=7mm,font=\tiny] {\csname light@caption@#1\endcsname};},
  light/.style={draw},
  on/.style={light,fill=yellow},
  off/.style={light,fill=gray!50},
  neutral/.style={light}
}

\title{Data Compression: LZ77}
\author{Fr\'ed\'eric Vogels}

\begin{document}

\begin{frame}
  \titlepage
\end{frame}

\begin{frame}
  \frametitle{Disclaimer}
  \begin{center}
    \Large
    Most examples assume we deal with text,
    but the principles apply to bytes in general
  \end{center}
\end{frame}

\begin{frame}
  \frametitle{Compression Algorithms}
  \begin{itemize}
    \item GQU
          \begin{itemize}
            \item Example algorithm
            \item Assumes pairs of bytes occur often
          \end{itemize}
    \item RLE
          \begin{itemize}
            \item Assumes long runs of the same value occur often
            \item TGA, PCX, fax machines
          \end{itemize}
    \item Huffman
          \begin{itemize}
            \item Assumes certain bytes occur more often
            \item Widely used (MP3, JPEG, zip, web, \dots)
          \end{itemize}
    \item<2-> LZ77/LZ78/LZSS/LZMA/\dots
          \begin{itemize}
            \item Assumes same series of bytes occur repeatedly
            \item Widely used (zip, 7z, rar, \dots)
          \end{itemize}
  \end{itemize}
\end{frame}

{
  \newcommand{\drawbox}[1]{\draw (#1.south west) rectangle (#1.north east);}
  \begin{frame}
    \frametitle{Example Input}
    \code[language={},font size=\small,width=\linewidth,frame=none]{singer.txt}
    \begin{tikzpicture}[overlay,remember picture,ref/.style={-latex,thick}]
      \visible<2->{
        \drawbox{inger1}
        \drawbox{inger2}
        \draw[ref] (inger2.east) to[bend right=30] (inger1.east);
      }

      \visible<3->{
        \drawbox{hoort1}
        \drawbox{hoort2}
        \draw[ref] (hoort2.east) to[bend right=30] (hoort1.east);
      }

      \visible<4->{
        \draw (singer1.north west) rectangle (inger2.south east);
        \drawbox{singer2}
        \draw[ref] (singer2.north) to[bend right=30] (inger2.east);
      }

      \visible<5->{
        \drawbox{naaimasjien1}
        \drawbox{naaimasjien2}
        \draw[ref] (naaimasjien2.north) to[bend right=30] (naaimasjien1.east);
      }

      \visible<6->{
        \drawbox{wat1}
        \drawbox{wat2}
        \draw[ref] (wat2.east) to[bend right=30] (wat1.east);
      }

      \visible<7->{
        \drawbox{jespers1}
        \drawbox{jespers2}
        \draw[ref] (jespers2.east) to[bend right=30] (jespers1.south);
      }

      \visible<8->{
        \drawbox{singer3}
        \drawbox{singer2}
        \draw[ref] (singer3.north east) to[bend right=10] (singer2.south);
      }

      \visible<9->{
        \drawbox{naaimasjien3}
        \drawbox{naaimasjien2}
        \draw[ref] (naaimasjien3.north east) to[bend right=10] (naaimasjien2.south);
      }
    \end{tikzpicture}
  \end{frame}
}

\begin{frame}
  \frametitle{Observations}
  \begin{itemize}
    \item Same words are repeated often
    \item Happens often in certain data files
          \begin{itemize}
            \item Source code (keywords, identifiers, etc.)
            \item Text files (word, pdf, etc.)
            \item XML
            \item \dots
          \end{itemize}
    \item More generally: certain series of bytes occur repeatedly
    \item We can use this to our advantage
  \end{itemize}
\end{frame}

\begin{frame}
  \frametitle{Basic Principle of LZ77}
  \begin{itemize}
    \item If same word has appeared before\dots
    \item \dots don't repeat it
    \item \dots but refer to the previous occurence
  \end{itemize}
\end{frame}

{
  \newcommand{\chunk}[3]{
    \draw[|-|] ($ (#1.south west) + (0,-0.1) $) -- ($ (#1.south east) + (0,-0.1) $);
    \draw[|->] ($ (#2.south west) + (0,-0.1) $) -- ($ (#2.south east) + (0,-0.1) $) node[midway,below,font=\tiny] {#3};
    \draw[-latex] ($ (#1.south) + (0,-0.1) $) -- ++(0,-0.5) -| ($ (#2.south west) + (0,-0.2) $);
  }
  \newcommand{\multichunk}[5]{
    \draw[|-|] ($ (#1.south west) + (0,-0.1) $) -- ($ (#2.south east) + (0,-0.1) $);
    \draw[|->] ($ (#3.south west) + (0,-0.1) $) -- ($ (#4.south east) + (0,-0.1) $) node[midway,below,font=\tiny] {#5};
    \draw[-latex] ($ (#1.south) ! 0.5 ! (#2.south) + (0,-0.1) $) -- ++(0,-0.5) -| ($ (#3.south west) + (0,-0.2) $);
  }
  \begin{frame}
    \frametitle{Conceptual Example}
    \structure{Data to be compressed}
    \begin{center} \tt
      \NODE{\alert<2>{foo}}{foo1}\NODE{\alert<3>{foo}}{foo2}\NODE{\alert<4>{bar}}{bar1}\NODE{\alert<5>{bar}}{bar2}\NODE{\alert<6>{foobarbar}}{foobarbar}\NODE{\alert<7>{foofoobarbarfoobarbar}}{last}
    \end{center}
    \begin{tikzpicture}[overlay,remember picture]
      \visible<3>{
        \chunk{foo2}{foo1}{3}
      }

      \visible<5>{
        \chunk{bar2}{bar1}{3}
      }

      \visible<6>{
        \multichunk{foobarbar}{foobarbar}{foo2}{bar2}{6}
      }

      \visible<7>{
        \multichunk{last}{last}{foo1}{foobarbar}{21}
      }
    \end{tikzpicture}
    \vskip5mm
    \structure{Compressed form}
    \begin{enumerate}
      \item<2-> {\tt foo}
      \item<3-> Repeat 3 letters starting from first letter
      \item<4-> {\tt bar}
      \item<5-> Repeat 3 letters starting from 6th letter
      \item<6-> Repeat 6 letters starting from 3rd letter
      \item<7-> Repeat 21 letters starting from first letter
    \end{enumerate}
  \end{frame}
}

\end{document}