\documentclass[]{pvm-exam}

\exam{
  academiejaar=2015--2016,
  examinator={F. Vogels},
  hulpmiddelen={Formularium},
  datum={TODO},
  beginuur={TODO},
}

\newcommand{\hex}[1]{\texttt{\bfseries #1}}

\newcommand{\lzsheet}[1]{
  \begin{center}
    \begin{tikzpicture}[box/.style={minimum width=1cm,minimum height=1.25cm,draw,font=\tt}]
      \foreach[count=\i] \byte/\given in {#1} {
        \tikzmath{
          int \col;
          int \row;
          real \x;
          real \y;
          \row = -int((\i - 1) / 8);
          \col = mod((\i - 1), 8);
          \x = \col * 1.25;
          \y = \row * 1.5;
        }

        \node[box] at (\x,\y) {
          % \edef\condparam{\yesno[\given]{always}{if solution}}
          \cond[\given]{\byte}
        };
      }
    \end{tikzpicture}
  \end{center}
}


\begin{document}

\begin{examguidelines}
  \begin{center}
    \textsc{\Huge richtlijnen}
  \end{center}
  \vskip1cm
  \Large
  \begin{itemize}
    \item Het examen is volledig schriftelijk.
    \item Volgende hulpmiddelen zijn toegelaten:
          \begin{itemize}
            \item Formularium
          \end{itemize}
    \item Enkel de antwoorden die je noteert op deze bladen
          tellen mee voor de beoordeling. Je kan geen extra bladen
          toevoegen, dus maak eerst indien nodig alles in klad om dan pas
          je finale antwoord op deze bladen te kopi\"eren.
    \item Bij vragen waar je een numerieke waarde voor moet uitrekenen
          staan er soms hints die je toelaten om na te gaan of je resultaat correct is.
  \end{itemize}
\end{examguidelines}



\begin{question}
  Comprimeer de string gebruik makende van LZ77.
  \begin{center} \tt
    ABCABCDABCDEDAB
  \end{center}
  Gebruik telkens 8 bits om de relatieve positie en de lengte van de matches voor te stellen.
  Noteer hieronder de resulterende bytes (hexadecimaal).
  \lzsheet{00/if solution,00/if solution,41/if solution,00/if solution,00/always,42/if solution,00/if solution,00/if solution,43/if solution,03/if solution,03/if solution,44/always,04/if solution,04/if solution,45/if solution,06/if solution,02/always,42/if solution}
\end{question}

\begin{question}
  Comprimeer de string gebruik makende van LZ77.
  \begin{center} \ttfamily
    XXYXXYZYXXYXWYXXYZ
  \end{center}
  Gebruik telkens 8 bits om de relatieve positie en de lengte van de matches voor te stellen.
  Noteer hieronder de resulterende bytes (hexadecimaal).
  \lzsheet{00/if solution,00/if solution,58/if solution,01/always,01/if solution,59/if solution,03/if solution,03/if solution,5a/if solution,05/if solution,04/if solution,58/if solution,00/always,00/if solution,57/if solution,06/if solution,04/if solution,5a/always}
\end{question}

\begin{question}
  Comprimeer de string gebruik makende van LZ77.
  \begin{center} \ttfamily
    ABABCBABDBABDABABA
  \end{center}
  Gebruik telkens 8 bits om de relatieve positie en de lengte van de matches voor te stellen.
  Noteer hieronder de resulterende bytes (hexadecimaal).
  \lzsheet{00/if solution,00/if solution,41/if solution,00/always,00/if solution,42/if solution,02/if solution,02/if solution,43/if solution,04/if solution,03/always,44/if solution,04/if solution,04/if solution,41/if solution,05/if solution,03/if solution,41/always}
\end{question}

\begin{question}
  Comprimeer de string gebruik makende van LZ77.
  \begin{center} \ttfamily
    AABCAABCDDABCADCAAB
  \end{center}
  Gebruik telkens 8 bits om de relatieve positie en de lengte van de matches voor te stellen.
  Noteer hieronder de resulterende bytes (hexadecimaal).
  \lzsheet{00/if solution,00/if solution,41/if solution,01/if solution,01/always,42/if solution,00/if solution,00/if solution,43/if solution,04/if solution,04/if solution,44/if solution,01/if solution,01/if solution,41/if solution,09/always,03/if solution,44/if solution,0c/if solution,03/if solution,42/always}
\end{question}


\begin{question}
  Decomprimeer de gegeven bytes gebruik makende van LZ77.
  \begin{center}
    \hex{00 00 46 00 00 4f 01 01 4c 03 02 46 04 04 46 05 01 46}
  \end{center}
  De relatieve positie en lengte van de matches zijn telkens voorgesteld d.m.v.~8~bits.
  Noteer hieronder het resultaat van decompressie (in tekstvorm).
  \vskip5mm
  \begin{center}
    \answerbox[width=6cm]{FOOLOOFLOOFFLF}
  \end{center}
\end{question}

\begin{question}
  Decomprimeer de gegeven bytes gebruik makende van LZ77.
  \begin{center}
    \hex{00 00 50 00 00 4f 00 00 4c 01 01 4f 05 02 50 07 03 4f}
  \end{center}
  De relatieve positie en lengte van de matches zijn telkens voorgesteld d.m.v.~8~bits.
  Noteer hieronder het resultaat van decompressie (in tekstvorm).
  \vskip5mm
  \begin{center}
    \answerbox[width=6cm]{BABARRABARBER}
  \end{center}
\end{question}

\begin{question}
  Decomprimeer de gegeven bytes gebruik makende van LZ77.
  \begin{center}
    \hex{00 00 50 00 00 4F 00 00 4C 01 01 4F 05 02 50 07 03 4F}
  \end{center}
  De relatieve positie en lengte van de matches zijn telkens voorgesteld d.m.v.~8~bits.
  Noteer hieronder het resultaat van decompressie (in tekstvorm).
  \vskip5mm
  \begin{center}
    \answerbox[width=6cm]{POLLOPOPOLLO}
  \end{center}
\end{question}

\begin{question}
  Decomprimeer de gegeven bytes gebruik makende van LZ77.
  \begin{center}
    \hex{00 00 44 00 00 49 00 00 50 01 01 49 05 02 44 07 02 49 00 00 44}
  \end{center}
  De relatieve positie en lengte van de matches zijn telkens voorgesteld d.m.v.~8~bits.
  Noteer hieronder het resultaat van decompressie (in tekstvorm).
  \vskip5mm
  \begin{center}
    \answerbox[width=6cm]{DIPPIDIDIPID}
  \end{center}
\end{question}


\end{document}