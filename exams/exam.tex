\usepackage{booktabs}

\exam{
  academiejaar=2015--2016,
  examinator={F. Vogels},
  hulpmiddelen={Formularium},
  datum={TODO},
  beginuur={TODO},
}

\newcommand{\cpp}{C++}

\begin{document}

\begin{examguidelines}
  \begin{center}
    \textsc{\Huge richtlijnen}
  \end{center}
  \vskip1cm% 
  \Large
  \begin{itemize}
    \item Het examen is volledig schriftelijk.
    \item Volgende hulpmiddelen zijn toegelaten:
          \begin{itemize}
            \item Formularium
          \end{itemize}
    \item Enkel de antwoorden die je noteert op deze bladen
          tellen mee voor de beoordeling. Je kan geen extra bladen
          toevoegen, dus maak eerst indien nodig alles in klad om dan pas
          je finale antwoord op deze bladen te kopi\"eren.
    \item Bij sommige vragen worden hints gegeven dewelke je kan gebruiken
          om na te gaan of je antwoorden kloppen.
    \item Gebruik geen \verb'#pragma once'.
  \end{itemize}
\end{examguidelines}

% \input{gen-interpretation-questions.tex}

\input{gen-translation-questions.tex}

% \begin{question}
  Gegeven de volgende frequenties, bouw een optimale Huffman-boom.
  \begin{center}
    \begin{tabular}{cc@{\hspace{1cm}}cc}
      \textbf{Letter} & \textbf{Frequentie} & \textbf{Letter} & \textbf{Frequentie} \\
      \toprule
      A & 50 & E & 25 \\
      B & 7 & F & 12 \\
      C & 44 & G & 19 \\
      D & 34 & H & 30 \\
      \bottomrule
    \end{tabular}
  \end{center}
  \vskip1cm
  \begin{solutionframe}[height=15cm]
\begin{solutionblock}
    \begin{center}
      \begin{tikzpicture}[node/.style={circle,black,draw,minimum size=8mm,inner sep=0mm,font=\tiny}]
        \node[node] (root) at (0,0) {};
        \node[node] (0) at ($ (root) + (-3,-1) $) {};
        \node[node] (00) at ($ (0) + (-1,-1) $) {C};
        \node[node] (01) at ($ (0) + (1,-1) $) {A};
        \node[node] (1) at ($ (root) + (3,-1) $) {};
        \node[node] (10) at ($ (1) + (-2,-1) $) {};
        \node[node] (100) at ($ (10) + (-1,-1) $) {E};
        \node[node] (101) at ($ (10) + (1,-1) $) {H};
        \node[node] (11) at ($ (1) + (2,-1) $) {};
        \node[node] (110) at ($ (11) + (-1,-1) $) {D};
        \node[node] (111) at ($ (11) + (1,-1) $) {};
        \node[node] (1110) at ($ (111) + (-1,-1) $) {G};
        \node[node] (1111) at ($ (111) + (1,-1) $) {};
        \node[node] (11110) at ($ (1111) + (-1,-1) $) {B};
        \node[node] (11111) at ($ (1111) + (1,-1) $) {F};

        \foreach \x/\y in {root/0,0/00,1/10,11/110,111/1110,1111/11110,10/100} {
          \draw (\x) -- (\y) node[sloped,midway,above,font=\tiny] {0};
        }

        \foreach \x/\y in {root/1,0/01,1/11,11/111,111/1111,1111/11111,10/101} {
          \draw (\x) -- (\y) node[sloped,midway,above,font=\tiny] {1};
        }
      \end{tikzpicture}
    \end{center}
\end{solutionblock}
  \end{solutionframe}
\end{question}

% \begin{question}
  Gegeven de volgende Huffman boom:
  \begin{center}
    \begin{tikzpicture}[node/.style={circle,black,draw,minimum size=8mm,inner sep=0mm,font=\tiny}]
      \node[node] (root) at (0,0) {};
      \node[node] (0) at ($ (root) + (-3,-1) $) {};
      \node[node] (00) at ($ (0) + (-1,-1) $) {C};
      \node[node] (01) at ($ (0) + (1,-1) $) {A};
      \node[node] (1) at ($ (root) + (3,-1) $) {};
      \node[node] (10) at ($ (1) + (-2,-1) $) {};
      \node[node] (100) at ($ (10) + (-1,-1) $) {E};
      \node[node] (101) at ($ (10) + (1,-1) $) {H};
      \node[node] (11) at ($ (1) + (2,-1) $) {};
      \node[node] (110) at ($ (11) + (-1,-1) $) {D};
      \node[node] (111) at ($ (11) + (1,-1) $) {};
      \node[node] (1110) at ($ (111) + (-1,-1) $) {G};
      \node[node] (1111) at ($ (111) + (1,-1) $) {};
      \node[node] (11110) at ($ (1111) + (-1,-1) $) {B};
      \node[node] (11111) at ($ (1111) + (1,-1) $) {F};

      \foreach \x/\y in {root/0,0/00,1/10,11/110,111/1110,1111/11110,10/100} {
        \draw (\x) -- (\y) node[sloped,midway,above,font=\tiny] {0};
      }

      \foreach \x/\y in {root/1,0/01,1/11,11/111,111/1111,1111/11111,10/101} {
        \draw (\x) -- (\y) node[sloped,midway,above,font=\tiny] {1};
      }
    \end{tikzpicture}
  \end{center}
  Encodeer de volgende string:
  \begin{center} \tt
    GDBAAFGDBAAF
  \end{center}
  Noteer hieronder de verkregen bits:
  \answersheet{1,1,1,0,1,1,0,1,1,1,1,0,0,1,0,1,1,1,1,1,1,1,1,1,0,1,1,0,1,1,1,1,0,0,1,0,1,1,1,1,1,1}
\end{question}

% \begin{question}
  Gegeven de volgende Huffman boom:
  \begin{center}
    \begin{tikzpicture}[node/.style={circle,black,draw,minimum size=8mm,inner sep=0mm,font=\tiny}]
      \node[node] (root) at (0,0) {};
      \node[node] (0) at ($ (root) + (-3,-1) $) {};
      \node[node] (00) at ($ (0) + (-1,-1) $) {C};
      \node[node] (01) at ($ (0) + (1,-1) $) {A};
      \node[node] (1) at ($ (root) + (3,-1) $) {};
      \node[node] (10) at ($ (1) + (-2,-1) $) {};
      \node[node] (100) at ($ (10) + (-1,-1) $) {E};
      \node[node] (101) at ($ (10) + (1,-1) $) {H};
      \node[node] (11) at ($ (1) + (2,-1) $) {};
      \node[node] (110) at ($ (11) + (-1,-1) $) {D};
      \node[node] (111) at ($ (11) + (1,-1) $) {};
      \node[node] (1110) at ($ (111) + (-1,-1) $) {G};
      \node[node] (1111) at ($ (111) + (1,-1) $) {};
      \node[node] (11110) at ($ (1111) + (-1,-1) $) {B};
      \node[node] (11111) at ($ (1111) + (1,-1) $) {F};

      \foreach \x/\y in {root/0,0/00,1/10,11/110,111/1110,1111/11110,10/100} {
        \draw (\x) -- (\y) node[sloped,midway,above,font=\tiny] {0};
      }

      \foreach \x/\y in {root/1,0/01,1/11,11/111,111/1111,1111/11111,10/101} {
        \draw (\x) -- (\y) node[sloped,midway,above,font=\tiny] {1};
      }
    \end{tikzpicture}
  \end{center}
  Decodeer de volgende bits:
  \begin{center} \tt
    011111000110101111011111100
  \end{center}
  Noteer hieronder de verkregen letters:
  \answersheet{A,B,C,D,H,G,F,E}
\end{question}


% \begin{question}
  Comprimeer de string gebruik makende van LZ77.
  \begin{center} \tt
    ABCABCDABCDEDAB
  \end{center}
  Gebruik telkens 8 bits om de relatieve positie en de lengte van de matches voor te stellen.
  Noteer hieronder de resulterende bytes (hexadecimaal).
  \lzsheet{00/if solution,00/if solution,41/if solution,00/if solution,00/always,42/if solution,00/if solution,00/if solution,43/if solution,03/if solution,03/if solution,44/always,04/if solution,04/if solution,45/if solution,06/if solution,02/always,42/if solution}
\end{question}

% \begin{question}
  Comprimeer de string gebruik makende van LZ77.
  \begin{center} \ttfamily
    XXYXXYZYXXYXWYXXYZ
  \end{center}
  Gebruik telkens 8 bits om de relatieve positie en de lengte van de matches voor te stellen.
  Noteer hieronder de resulterende bytes (hexadecimaal).
  \lzsheet{00/if solution,00/if solution,58/if solution,01/always,01/if solution,59/if solution,03/if solution,03/if solution,5a/if solution,05/if solution,04/if solution,58/if solution,00/always,00/if solution,57/if solution,06/if solution,04/if solution,5a/always}
\end{question}

% \begin{question}
  Comprimeer de string gebruik makende van LZ77.
  \begin{center} \ttfamily
    ABABCBABDBABDABABA
  \end{center}
  Gebruik telkens 8 bits om de relatieve positie en de lengte van de matches voor te stellen.
  Noteer hieronder de resulterende bytes (hexadecimaal).
  \lzsheet{00/if solution,00/if solution,41/if solution,00/always,00/if solution,42/if solution,02/if solution,02/if solution,43/if solution,04/if solution,03/always,44/if solution,04/if solution,04/if solution,41/if solution,05/if solution,03/if solution,41/always}
\end{question}

% \begin{question}
  Comprimeer de string gebruik makende van LZ77.
  \begin{center} \ttfamily
    AABCAABCDDABCADCAAB
  \end{center}
  Gebruik telkens 8 bits om de relatieve positie en de lengte van de matches voor te stellen.
  Noteer hieronder de resulterende bytes (hexadecimaal).
  \lzsheet{00/if solution,00/if solution,41/if solution,01/if solution,01/always,42/if solution,00/if solution,00/if solution,43/if solution,04/if solution,04/if solution,44/if solution,01/if solution,01/if solution,41/if solution,09/always,03/if solution,44/if solution,0c/if solution,03/if solution,42/always}
\end{question}


% \begin{question}
  Decomprimeer de gegeven bytes gebruik makende van LZ77.
  \begin{center}
    \hex{00 00 46 00 00 4f 01 01 4c 03 02 46 04 04 46 05 01 46}
  \end{center}
  De relatieve positie en lengte van de matches zijn telkens voorgesteld d.m.v.~8~bits.
  Noteer hieronder het resultaat van decompressie (in tekstvorm).
  \vskip5mm
  \begin{center}
    \answerbox[width=6cm]{FOOLOOFLOOFFLF}
  \end{center}
\end{question}

% \begin{question}
  Decomprimeer de gegeven bytes gebruik makende van LZ77.
  \begin{center}
    \hex{00 00 50 00 00 4f 00 00 4c 01 01 4f 05 02 50 07 03 4f}
  \end{center}
  De relatieve positie en lengte van de matches zijn telkens voorgesteld d.m.v.~8~bits.
  Noteer hieronder het resultaat van decompressie (in tekstvorm).
  \vskip5mm
  \begin{center}
    \answerbox[width=6cm]{BABARRABARBER}
  \end{center}
\end{question}

% \begin{question}
  Decomprimeer de gegeven bytes gebruik makende van LZ77.
  \begin{center}
    \hex{00 00 50 00 00 4F 00 00 4C 01 01 4F 05 02 50 07 03 4F}
  \end{center}
  De relatieve positie en lengte van de matches zijn telkens voorgesteld d.m.v.~8~bits.
  Noteer hieronder het resultaat van decompressie (in tekstvorm).
  \vskip5mm
  \begin{center}
    \answerbox[width=6cm]{POLLOPOPOLLO}
  \end{center}
\end{question}

% \begin{question}
  Decomprimeer de gegeven bytes gebruik makende van LZ77.
  \begin{center}
    \hex{00 00 44 00 00 49 00 00 50 01 01 49 05 02 44 07 02 49 00 00 44}
  \end{center}
  De relatieve positie en lengte van de matches zijn telkens voorgesteld d.m.v.~8~bits.
  Noteer hieronder het resultaat van decompressie (in tekstvorm).
  \vskip5mm
  \begin{center}
    \answerbox[width=6cm]{DIPPIDIDIPID}
  \end{center}
\end{question}


% \begin{question}
  Welke compressiealgoritmes werken het best op volgende inputs?
  Ga ervan uit dat we niet per byte, maar per nibble (4 bits) werken, m.a.w.~er
  zijn slechts 16 mogelijke waarden: \texttt{0123456789ABCDEF}.
  \begin{center}
    \begin{tabular}{ll}
      \toprule
      \textbf{Input} & \textbf{Algoritme} \\
      \midrule
      \parbox[c]{12cm}{
        \texttt{0123456789ABCDEF0123456789ABCDEF0123456789ABCDEF \\
                0123456789ABCDEF0123456789ABCDEF0123456789ABCDEF}} &%
      \parbox[c]{2cm}{
        \checkbox[unchecked,if solution] RLE \\
        \checkbox[unchecked,if solution] Huffman \\
        \checkbox[checked,if solution] LZ77
      } \\
      \midrule
      \textbf{Uitleg} \\[2mm]
      \multicolumn{2}{l}{%
        \parbox[t][5cm][l]{14cm}{%
          \ifsolution{RLE faalt omdat er geen reeksen gelijke tekens voorkomen in de input.
          Huffman faalt omdat elk teken even vaak voorkomt.
          LZ77 zal het best werken omdat hetzelfde patroon
          \texttt{0123456789ABCDEF} telkens herhaald wordt.}}} \\
      \bottomrule
    \end{tabular}
  \end{center}
\end{question}

% \begin{question}
  Welke compressiealgoritmes werken het best op volgende inputs?
  Ga ervan uit dat we niet per byte, maar per nibble (4 bits) werken, m.a.w.~er
  zijn slechts 16 mogelijke waarden: \texttt{0123456789ABCDEF}.
  \begin{center}
    \begin{tabular}{ll}
      \toprule
      \textbf{Input} & \textbf{Algoritme} \\
      \midrule
      \parbox[c]{12cm}{%
        \texttt{000000001111111122222222333333334444444455555555 \\
                66666666777777778888888899999999AAAAAAAABBBBBBBB \\
                }} &%
      \parbox[c]{2cm}{
        \checkbox[checked,if solution] RLE \\
        \checkbox[unchecked,if solution] Huffman \\
        \checkbox[unchecked,if solution] LZ77
      } \\
      \midrule
      \textbf{Uitleg} \\[2mm]
      \multicolumn{2}{l}{%
        \parbox[c][5cm][l]{14cm}{%
          \ifsolution{RLE werkt het best omwille van de opeenvolgende gelijke tekens.
          Huffman zal slecht werken omdat elk teken ongeveer even vaak voorkomt.
          LZ77 zal lokaal wat werken, maar vermoedelijk weinig bits besparen.
          }}} \\
      \bottomrule
    \end{tabular}
  \end{center}
\end{question}

% \begin{question}
  Welke compressiealgoritmes werken het best op volgende inputs?
  Ga ervan uit dat we niet per byte, maar per nibble (4 bits) werken, m.a.w.~er
  zijn slechts 16 mogelijke waarden: \texttt{0123456789ABCDEF}.
  \begin{center}
    \begin{tabular}{ll}
      \toprule
      \textbf{Input} & \textbf{Algoritme} \\
      \midrule
      \parbox[c]{12cm}{%
        \texttt{AAAAAAAAAAAAAAAAAAAAAAAAAAAAAAAAAAAAAAAAAAAAAAAAAAAAAAA}} &%
      \parbox[c]{2cm}{
        \checkbox[checked,if solution] RLE \\
        \checkbox[checked,if solution] Huffman \\
        \checkbox[checked,if solution] LZ77
      } \\
      \midrule
      \textbf{Uitleg} \\[2mm]
      \multicolumn{2}{l}{%
        \parbox[c][5cm][l]{14cm}{%
          \ifsolution{RLE werkt uitstekend. Huffman zal 1 bit per teken kunnen toekennen,
          maar voor lange reeksen zullen RLE en LZ77 meer winst opleveren. Hoe langer de reeks,
          hoe meer winst LZ77 zal opleveren.
          }}} \\
      \bottomrule
    \end{tabular}
  \end{center}
\end{question}


\end{document}

%%% Local Variables:
%%% mode: latex
%%% TeX-master: "with-solutions"
%%% End:
