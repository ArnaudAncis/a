% \def\solution{}
\documentclass{ucll-exam}

\exam{
  academiejaar=2015--2016,
  opleiding={Bachelor in de Toegepaste Informatica},
  fase={1},
  examinator={F. Vogels},
  opo={MBI66A -- Programmeren voor Multimedia},
  ola={MBI66a -- Programmeren voor Multimedia},
  activiteit={Examen},
  hulpmiddelen={Rekenmachine, formularium},
  datum={TODO},
  beginuur={TODO},
  duur={3 uur}
}


\newcommand{\lzsheet}[1]{
  \begin{center}
    \begin{tikzpicture}[box/.style={minimum width=1cm,minimum height=1.25cm,draw,font=\tt}]
      \foreach[count=\i] \byte/\given in {#1} {
        \tikzmath{
          int \col;
          int \row;
          real \x;
          real \y;
          \row = -int((\i - 1) / 8);
          \col = mod((\i - 1), 8);
          \x = \col * 1.25;
          \y = \row * 1.5;
        }

        \node[box] at (\x,\y) {\showif[show=\given]{\byte}};
      }
    \end{tikzpicture}
  \end{center}
}


\begin{document}

\newgeometry{includeheadfoot,headheight=5cm,top=2cm,bottom=1.5cm}
\vspace*{\fill}
\begin{center}
  \begin{framed}
    \begin{minipage}{.8\linewidth}
      \begin{center}
        \textsc{\Huge richtlijnen}
      \end{center}
      \vskip1cm
      \Large
      \begin{itemize}
        \item Het examen is volledig schriftelijk.
        \item Volgende hulpmiddelen zijn toegelaten:
              \begin{itemize}
                \item Formularium
              \end{itemize}
        \item Enkel de antwoorden die je noteert op deze bladen
              tellen mee voor de beoordeling. Je kan geen extra bladen
              toevoegen, dus maak eerst indien nodig alles in klad om dan pas
              je finale antwoord op deze bladen te kopi\"eren.
        \item Bij vragen waar je een numerieke waarde voor moet uitrekenen
              staan er soms hints die je toelaten om na te gaan of je resultaat correct is.
      \end{itemize}
    \end{minipage}
  \end{framed}
\end{center}
\vspace*{\fill}\vspace*{\fill}
\restoregeometry
\clearpage


%%% Local Variables:
%%% mode: latex
%%% TeX-master: "exam"
%%% End:


\begin{question}
  Comprimeer de string gebruik makende van LZ77.
  \begin{center} \tt
    ABCABCDABCDEDAB
  \end{center}
  Gebruik telkens 8 bits om de relatieve positie en de lengte van de matches voor te stellen.
  Noteer hieronder de resulterende bytes (hexadecimaal).
  \lzsheet{00/solution,00/solution,41/solution,00/solution,00/always,42/solution,00/solution,00/solution,43/solution,03/solution,03/solution,44/always,04/solution,04/solution,45/solution,06/solution,02/always,42/solution}
\end{question}

\vfil

\begin{question}
  Comprimeer de string gebruik makende van LZ77.
  \begin{center} \tt
    XXYXXYZYXXYXWYXXYZ
  \end{center}
  Gebruik telkens 8 bits om de relatieve positie en de lengte van de matches voor te stellen.
  Noteer hieronder de resulterende bytes (hexadecimaal).
  \lzsheet{00/solution,00/solution,58/solution,01/always,01/solution,59/solution,03/solution,03/solution,5a/solution,05/solution,04/solution,58/solution,00/always,00/solution,57/solution,06/solution,04/solution,5a/always}
\end{question}

\begin{question}
  Comprimeer de string gebruik makende van LZ77.
  \begin{center} \tt
    ABABCBABDBABDABABA
  \end{center}
  Gebruik telkens 8 bits om de relatieve positie en de lengte van de matches voor te stellen.
  Noteer hieronder de resulterende bytes (hexadecimaal).
  \lzsheet{00/solution,00/solution,41/solution,00/always,00/solution,42/solution,02/solution,02/solution,43/solution,04/solution,03/always,44/solution,04/solution,04/solution,41/solution,05/solution,03/solution,41/always}
\end{question}

\vfil

\begin{question}
  Comprimeer de string gebruik makende van LZ77.
  \begin{center} \tt
    AABCAABCDDABCADCAAB
  \end{center}
  Gebruik telkens 8 bits om de relatieve positie en de lengte van de matches voor te stellen.
  Noteer hieronder de resulterende bytes (hexadecimaal).
  \lzsheet{00/solution,00/solution,41/solution,01/solution,01/always,42/solution,00/solution,00/solution,43/solution,04/solution,04/solution,44/solution,01/solution,01/solution,41/solution,09/always,03/solution,44/solution,0c/solution,03/solution,42/always}
\end{question}



\end{document}